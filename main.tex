%% Copernicus Publications Manuscript Preparation Template for LaTeX Submissions
%% ---------------------------------
%% This template should be used for copernicus.cls
%% The class file and some style files are bundled in the Copernicus Latex Package, which can be downloaded from the different journal webpages.
%% For further assistance please contact Copernicus Publications at: production@copernicus.org
%% https://publications.copernicus.org/for_authors/manuscript_preparation.html


%% Please use the following documentclass and journal abbreviations for preprints and final revised papers.

%% 2-column papers and preprints
\documentclass[journal abbreviation, manuscript]{copernicus}
%\documentclass[acp]{copernicus} % for some reason you have to run this a few times to get the abstract to appear


%% Journal abbreviations (please use the same for preprints and final revised papers)


% Advances in Geosciences (adgeo)
% Advances in Radio Science (ars)
% Advances in Science and Research (asr)
% Advances in Statistical Climatology, Meteorology and Oceanography (ascmo)
% Aerosol Research (ar)
% Annales Geophysicae (angeo)
% Archives Animal Breeding (aab)
% Atmospheric Chemistry and Physics (acp)
% Atmospheric Measurement Techniques (amt)
% Biogeosciences (bg)
% Climate of the Past (cp)
% DEUQUA Special Publications (deuquasp)
% Earth Surface Dynamics (esurf)
% Earth System Dynamics (esd)
% Earth System Science Data (essd)
% E&G Quaternary Science Journal (egqsj)
% EGUsphere (egusphere) | This is only for EGUsphere preprints submitted without relation to an EGU journal.
% European Journal of Mineralogy (ejm)
% Fossil Record (fr)
% Geochronology (gchron)
% Geographica Helvetica (gh)
% Geoscience Communication (gc)
% Geoscientific Instrumentation, Methods and Data Systems (gi)
% Geoscientific Model Development (gmd)
% History of Geo- and Space Sciences (hgss)
% Hydrology and Earth System Sciences (hess)
% Journal of Bone and Joint Infection (jbji)
% Journal of Micropalaeontology (jm)
% Journal of Sensors and Sensor Systems (jsss)
% Magnetic Resonance (mr)
% Mechanical Sciences (ms)
% Natural Hazards and Earth System Sciences (nhess)
% Nonlinear Processes in Geophysics (npg)
% Ocean Science (os)
% Polarforschung - Journal of the German Society for Polar Research (polf)
% Primate Biology (pb)
% Proceedings of the International Association of Hydrological Sciences (piahs)
% Safety of Nuclear Waste Disposal (sand)
% Scientific Drilling (sd)
% SOIL (soil)
% Solid Earth (se)
% State of the Planet (sp)
% The Cryosphere (tc)
% Weather and Climate Dynamics (wcd)
% Web Ecology (we)
% Wind Energy Science (wes)


%% \usepackage commands included in the copernicus.cls:
%\usepackage[german, english]{babel}
%\usepackage{tabularx}
%\usepackage{cancel}
%\usepackage{multirow}
%\usepackage{supertabular}
%\usepackage{algorithmic}
%\usepackage{algorithm}
%\usepackage{amsthm}
%\usepackage{float}
%\usepackage{subfig}
%\usepackage{rotating}
\usepackage{xcolor}
\usepackage{soul}
\usepackage{listings}
\usepackage{booktabs}


\begin{document}

\title{Idealized Particle-Resolved Large-Eddy Simulations to Evaluate the Impact of Emissions Spatial Heterogeneity on CCN Activity}

\Author[1]{Samuel}{Frederick}
\Author[2]{Matin}{Mohebalhojeh}
\Author[1,2]{Jeffrey}{Curtis}
\Author[2]{Matthew}{West}
\Author[1][nriemer@illinois.edu]{Nicole}{Riemer} %% correspondence author


\affil[1]{Department of Climate, Meteorology, and Atmospheric Sciences, University of Illinois Urbana--Champaign, 105 S Gregory St., Urbana, IL 61801, USA}
\affil[2]{Department of Mechanical Science and Engineering, University of Illinois Urbana--Champaign, 1206 W. Green St., Urbana, IL, 61801,USA}

%% If authors contributed equally, please mark the respective author names with an asterisk, e.g. "\Author[2,*]{Anton}{Smith}" and "\Author[3,*]{Bradley}{Miller}" and add a further affiliation: "\affil[*]{These authors contributed equally to this work.}".

\runningtitle{Impacts of Emissions SH on CCN activity}

\runningauthor{Frederick}

\received{}
\pubdiscuss{} %% only important for two-stage journals
\revised{}
\accepted{}
\published{}

%% These dates will be inserted by Copernicus Publications during the typesetting process.

\firstpage{1}

\maketitle

\begin{abstract}
Aerosol-cloud interactions remain a large source of uncertainty in global climate models (GCMs) due to complex, nonlinear processes that alter aerosol properties and the inability to represent the full compositional complexity of aerosol populations within large-scale modeling frameworks. The spatial resolution of GCMs is often coarser than the scale of the spatially varying emissions in the modeled geographic region. This results in diffuse, uniform concentration fields of primary aerosol and gas-phase species instead of spatially heterogeneous concentrations. Aerosol processes such as gas-particle partitioning and coagulation are concentration-dependent in a non-linear manner, and thus the representation of spatially heterogeneous emissions impacts aerosol aging and properties. This includes climate-relevant quantities key to aerosol-cloud interactions including particle hygroscopicity and cloud condensation nuclei (CCN) activity. We investigate the impact of emissions spatial heterogeneity on aerosol properties including CCN activity via a series of first-of-a-kind particle-resolved large-eddy simulations with the modeling framework WRF-PartMC-MOSAIC-LES. CCN concentrations within the planetary boundary layer (PBL) are compared across numerous scenarios ranging in emissions spatial heterogeneity. We find that CCN concentrations at low supersaturations ($S=0.1\mbox{--}0.3\%$) increase in the upper PBL by up to 25\%
for emissions scenarios with high spatial heterogeneity when compared to a uniform emissions base case.
\end{abstract}


%\copyrightstatement{TEXT} %% This section is optional and can be used for copyright transfers.


\introduction  %% \introduction[modified heading if necessary]


Aerosols are known to impose a net negative radiative forcing, however considerable uncertainty persists in the representation of radiative forcing due to aerosol-cloud interactions (ACI) \citep{ipcc_report_2021}. Advancements in computing power have allowed modelers to investigate key contributing factors which alter the intensity of radiative forcing due to ACI such as spatial resolution \citep{ma_how_2015}. Simultaneously, improvements in computational power have advanced the representation of aerosols in climate models through inclusion of more complex chemical mechanisms (\hl{citation?}) and detailed aerosol treatments such as the Community Aerosol and Radiation Model for Atmospheres (CARMA) , which is a 40 bin sectional model used by the Community Earth System Model 2 (CESM2) \citep{tilmes_description_2023} and has shown improvements in representing aerosol properties when compared to the 4 bin Modal Aerosol Model (MAM4). While these two key structural components to models---the treatment of aerosols and the effect of spatial resolution---have improved representation of aerosols and their radiative forcing, these aspects have yet to be explored in tandem to investigate the role of spatial heterogeneities typically below the grid scale of current climate models alongside detailed aerosol treatments in modifying the depiction of climate-relevant aerosol properties key to ACI such as cloud condensation nuclei (CCN) activity.

%Aerosol concentrations and properties vary at scales ranging from global and regional variability down to local and hyper-local spatial heterogeneity near emissions sources. Numerous lines of evidence point to the multi-scale spatial variability of aerosols. At the global-scale, satellite remote sensing with platforms such as MODIS have shown large-scale variability in bulk quantities including aerosol optical depth (\hl{cite}). At the regional scale, field campaigns have shown considerable variability in aerosol properties such as concentration, size distributions, CCN concentrations, and composition \citep{fast_using_2022}. Local in-situ measurements of aerosol properties are highly dependent on proximity to emission sources such as vehicular combustion, and past studies have shown that aerosol abundance and composition vary on the scale of 10s to 100s of meters downwind of sources due to non-linear processes such as coagulation \citep{zhu_study_2002}. Indeed, hyper-local heterogeneity in aerosol properties frustrates traditional comparison between point-source measurements and grid-cell averaged model quantities, and distributed measurement networks such as the Portable Optical Particle Spectrometer network in the Southern Great Plains (POPSnet-SGP) campaign aim to elucidate climate model uncertainty associated with sub-grid scale aerosol spatial heterogeneity \citep{asher_novel_2022}. 

%\hl{Could also note that aerosol heterogeneities are coupled to other sources of heterogeneity by inducing feedbacks on, for instance, heterogeneous surface properties such as radiative heating and soil moisture, which have been shown to induce secondary circulations (Lee et al. 2019, Tian et al. 2022)}

Aerosol-aware climate models typically contain a sub-model which governs the aerosol representation and associated processes. Given computational constraints, the aerosol treatment is often highly simplified with regard to aerosol compositional diversity. For instance, E3SM uses the modal aerosol model MAM4 which contains four lognormally distributed modes \cite{golaz_doe_2022}. Such a treatment constrains the diversity of the aerosol population to four internally-mixed modes (i.e., all particles within a mode are compositionally identical). As a result, modal and sectional aerosol treatments underrepresent the true compositional complexity of an aerosol population in which each particle ages independently. Particle-resolved aerosol models allow representation of the full compositional diversity of an aerosol population via a set of computational particles which are allowed to compositionally vary and age independently. Particle resolved models such as the Particle Monte Carlo model (PartMC) have been used extensively to investigate the sensitivity of CCN activity to the composition of emitted aerosol particles \cite{fierce_when_2013}, aging timescales due to condensation and coagulation for carbonaceous CCN \citep{fierce_explaining_2015}, and the sensitivity of CCN estimates to mixing state (i.e., the compositional diversity within and across aerosols) \citep{ching_metrics_2017}. Furthermore, given that a particle-resolved model can fully represent aerosol composition space, PartMC has been used to estimate error in CCN activity for coarser aerosol treatments by comparing depicted CCN concentrations against compositionally-averaged output reminiscent of a sectional or modal model \citep{zaveri_particle-resolved_2010, ching_metrics_2017}. Furthermore, direct comparison of CCN activity depicted by PartMC and MAM4 has shown considerable divergence, especially in polluted regions where high rates of coagulation and gas-particle partitioning amplify model disagreement \citep{fierce_quantifying_2024}. 

In addition to the treatment of aerosols, the depiction of spatial heterogeneity---including that of surface properties, emissions of gas phase species and primary aerosol, and heterogeneous distributions of their resulting plumes---influences particle aging and associated properties including CCN activity. Present aerosol-aware models at regional and global scales possess insufficient resolution to capture the full spatial heterogeneity of aerosols and the emissions of primary aerosols and gas phase precursors. In turn, these models depict artificially dilute and uniform concentrations within grid cells. This alters the representation of concentration-dependent, non-linear aerosol processes such as coagulation and gas particle partitioning. Past studies have shown that climate-relevant aerosol properties such as aerosol optical properties \citep{gustafson_jr_downscaling_2011} and CCN activity \citep{weigum_effect_2016} are highly sensitive to the model’s grid resolution with a large contribution of sub-grid scale variability resulting from the spatially varying pattern of emissions \citep{qian_investigation_2010}. 

%Given large uncertainties in the effective radiative forcing resulting from aerosol-cloud interactions \citep{ipcc_report_2021}, constraining model estimates of CCN activity remains an important focus. 

Past studies evaluating the sub-grid variability of aerosol properties often compare outputs from models with coarse resolution typical of global climate models ($\sim 50$--$100$ km) against higher resolution model outputs with resolution on the order of $\sim 1$--$10$ km 
\citep{qian_investigation_2010, gustafson_jr_downscaling_2011, weigum_effect_2016, crippa_impact_2017, lin_quantification_2017}. While such improvements in resolution better resolve heterogeneities in emissions, there still exists considerable unresolved spatial heterogeneity in the sub-kilometer scale. In addition, such modeling studies do not explicitly resolve the scales of turbulent transport in the boundary layer, instead fully parameterizing turbulence via Reynolds-averaged Navier Stokes. 

%Furthermore, these models use simplified representations of aerosols such as modal or sectional treatments. Comparing these aerosol treatments against particle-resolved models which represent the broad compositional complexity of aerosols, past studies have shown that simplified aerosol representations impact modeled aerosol properties including CCN activity \citep{zaveri_particle-resolved_2010, ching_metrics_2017, fierce_quantifying_2024}.

The evolution of emission plumes containing gas phase compounds and aerosols is highly dependent on turbulent mixing at fine spatial scales and the proximity of reactive species. A considerable body of literature has investigated the role of turbulence-chemistry interactions and chemical segregation on gas phase reactions in the planetary boundary layer via the use of large-eddy simulations (LES). \citet{brasseur_segregation_2023} review recent applications of LES to investigate chemical segregation and turbulence chemistry interactions in a variety of spatially heterogeneous geographic regions. Past studies tend to focus on the oxidation of highly reactive volatile organic compounds (VOCs) such as isoprene and have shown that spatially heterogeneous emissions contribute to the chemical segregation between reactive gas phase species in the boundary layer \citep{ouwersloot_segregation_2011, kaser_chemistry-turbulence_2015}. Note, however, that these studies do not model aerosols, although the coupling between the gas phase and aerosols through gas-particle partitioning suggests chemical segregation due to the spatial heterogeneity of emissions likely influences the aerosol state. 
%\hl{Also note Jeff Pierce's work}


Recently, numerous turbulence-resolving frameworks have been coupled to aerosol models including the use of the Sectional Aerosol model for Large Scale Applications (SALSA) \citep{kokkola_salsa_2008} alongside UCLALES \citep{tonttila_uclalessalsa_2017} and the Parallelized Large-Eddy Simulation Model (PALM) \citep{kurppa_implementation_2019} as well as the coupling between the modal model M7 \citep{vignati_m7_2004} and the Dutch Atmospheric Large-Eddy Simulation model (DALES) \citep{de_bruine_explicit_2019}.
%Both models have been used to investigate aerosol-cloud interactions and compare model outputs against field campaign measurements. 
Although these models have high-resolution transport schemes, they each possess relatively coarse-resolution aerosol treatments. For instance, UCLALES-SALSA uses a 10-bin sectional treatment while DALES implements a modified version of the seven-mode M7 model to allow two additional hydrometeor modes. To our knowledge, no aerosol-aware transport model has yet to leverage a high-resolution particle resolved aerosol treatment alongside turbulence resolving transport frameworks such as LES. 

%\hl{I don't really consider LES models that couple microphysics schemes (such as the microphysics scheme of Feingold 1996) since they simply parameterize the aerosols and number of CCN but there are more of such models in existence than purely aerosol-aware LES.}

%\begin{itemize}
%\item SAM + modal Aitken mode aerosol microphysics scheme (Wyant et al. 2022)
%\item PNNL-LES + 2D binned microphysics (Ovchinnikov and Easter 2010)
%\end{itemize}

%\hl{This is mostly a synopsis of stuff I've just discussed so could removed if necessary.} 
Past research has established a clear link between the spatial heterogeneity of both gas phase and aerosol emissions, the processes by which the aerosol age, and their resulting climate-relevant properties including CCN activity which contribute to indirect radiative forcing. Simultaneously, there exists large uncertainty in the radiative forcing due to aerosols which results in part from the coarse representation of both aerosols and their transport in global scale models and their coupling with non-linear processes that occur at the sub-grid scale such as coagulation and turbulence-chemistry interactions. Past research has led to the development of detailed aerosol model treatments and transport representations, however there has yet to be a direct coupling between particle-resolved aerosol models and turbulence-resolving transport models for use in quantifying the effects of emissions spatial heterogeneity on the aerosol state and CCN activity.

The aim of this work is to conduct a process-level analysis of the the complex coupling between the spatial heterogeneity of surface emissions (including both gas compounds and primary aerosol), aerosol aging processes, and the resulting impact on aerosol properties including CCN activity via a set of first-of-a-kind particle-resolved LES. This establishes a high resolution aerosol-transport model framework in both explicit representation of turbulent transport as well as aerosol composition, properties, and aging.

This paper is organized in the following manner. Section 2 presents the modeling framework used in this study, WRF-PartMC-MOSAIC-LES, alongside a description of numerous emissions scenarios ranging in spatial heterogeneity. Section 3 discusses results of simulation runs, and a description of changes to the aerosol size distribution, composition, hygroscopicity, and CCN activity across emissions scenarios are discussed. We conclude with remarks on the implications of this study, limitations stemming from its idealized nature, and future work. 



\section{Methods}

\subsection{WRF-PartMC-MOSAIC-LES}

The aerosol-transport model WRF-PartMC-MOSAIC-LES is a coupling of numerous sub-models responsible for transport, the aerosol representation, and multiphase chemistry. WRF-PartMC-MOSAIC-LES is an extension of the aerosol-transport model WRF-PartMC, which was developed by \citep{curtis_single-column_2017} for particle diffusion and recently extended to include advection \citep{gmd-17-8399-2024}. WRF-PartMC couples the Weather Research and Forecasting model (WRF) \citep{skamarock_description_2008} and the particle-resolved aerosol model Particle Monte Carlo (PartMC) \citep{riemer_simulating_2009}. PartMC represents a population of aerosol particles using an ensemble of computational particles, each with appropriate weighting (i.e., particle multiplicity) to represent the diversity of particle sizes and composition apparent in actual aerosol populations. The composition of each computational particle is allowed to vary as particles age. PartMC is a box model, meaning that the relative position of particles within computational grid cells is not tracked. Instead, the transport of particles into and out of grid cells is handled by a stochastic advection algorithm housed in the interface between PartMC and the dynamical model WRF \citep{gmd-17-8399-2024}. \hl{Also talk about how PartMC has been used to describe complex mixing state?} WRF-PartMC has been used \hl{in a 1-D setting to resolve vertical gradients in aerosol composition and mixing state and} at regional scales to \hl{evaluate the impact of aerosol representation on particle mixing state and climate-relevant properties including CCN activity -- this is briefly mentioned at the end of Jeff's 2024 paper, maybe cite something else?}. 

LES models explicitly resolve large scales of turbulent motion; however, they must parameterize eddies which cannot be resolved below the grid resolution on the order of 10--100 m and the down-gradient tendency of turbulent kinetic energy (TKE) to be transferred from large to small scales, ultimately dissipating as heat energy. This requires use of eddy diffusivity and eddy viscosity parameterizations, and WRF-PartMC-MOSAIC-LES utilizes Deardorff's TKE scheme \citep{deardorff_stratocumulus-capped_1980}.

Both gas-phase and aerosol chemistry are represented using the Model for Simulating Aerosol Interactions and Chemistry (MOSAIC) \citep{zaveri_model_2008}. MOSAIC is comprised of numerous sub-models, including the Carbon Bond Mechanism version Z (CBM-Z) which solves gas phase chemistry \citep{zaveri_new_1999}. Phase-dependent partitioning of aerosol species is handled by the Multicomponent Equilibrium Solver for Aerosols (MESA) \citep{zaveri_computationally_2005}. Activity coefficients of electrolytes are parameterized via the multicomponent Taylor expansion method (MTEM) \citep{zaveri_new_2005}. In order to solve the numerically stiff set of solid-liquid phase reactions governing thermodynamic equilibrium, MOSAIC utilizes the adaptive step time-split Euler method (ASTEM) \citep{zaveri_model_2008}. MOSAIC models aerosol chemistry for both inorganic and organic compounds such as nitrate, ammonium, sulfate, black carbon (BC), and a limited set of secondary organic aerosol (SOA) species. 

\subsection{Computational domain setup}

The computational domain has a horizontal extent of 10 km in both $x$- and $y$-axes, with horizontal grid spacing of 100 m. In the vertical, the domain extends to 2 km and is represented with 200 vertical levels. WRF uses an $\eta$ vertical coordinate system and for LES runs, vertical levels are linearly spaced in pressure. This results in an effective vertical grid spacing of approximately 10 m. Simulations are configured to begin on the Vernal Equinox at 09:00 local time and conclude at 15:00 for a total duration of 6 hours in order to balance photolysis rates throughout simulations. Each grid cell is initialized with 100 computational particles, resulting in 100 million total particles. 

The surface of the domain is characterized by a uniform, flat surface absent of topographical features or land-use variations. Note that WRF-PartMC-MOSAIC-LES is not coupled to one of WRF's radiation sub-models. Instead,  
MOSAIC utilizes idealized parameterizations to determine photolysis rates based on the solar zenith angle. Due to the lack of a radiation sub-model, surface heating is imposed uniformly across the domain using a constant rate of 0.24 K m$^{-1}$ s$^{-1}$. 

Both aerosol and gas phase initial conditions and emissions are chosen to represent species and concentrations typical of an urban plume and are adopted from \citet{riemer_simulating_2009}. Initial concentrations and emission rates were adapted from the 1987 Southern California Air Quality Study (SCAQS) during which measurements of gas phase species and particulate matter mass concentrations were collected at multiple sites across the Los Angeles basin \citep{zaveri_model_2008}. Table \ref{table:gas_emiss_ics} contains initial concentrations and emission rates for gas phase species. Table \ref{table:aero_emiss_ics} includes aerosol initial conditions and emission rates organized by aerosol modes. Initially, the aerosol is an equal mixture of ammonium sulfate and primary organic aerosol (POA). The three emission modes including cooking and vehicular combustion are varied mixtures of POA and BC. To allow for simulation spin-up during which time the convective boundary layer fully develops, all emission rates are set to zero during the first hour of simulations. Subsequently, emitted compounds are released at the surface at constant rates indicated by Tables \ref{table:gas_emiss_ics} and \ref{table:aero_emiss_ics} for the remainder of simulations.


\begin{table}[!t]
\centering
\caption{Gas phase emissions and initial conditions. Table adapted from \citet{riemer_simulating_2009} with permission.}
\begin{tabular*}{\linewidth}{@{\extracolsep{\fill}} lccr}
\\[-2ex]\hline 
     \hline \\[-2ex] Species & Symbol & Initial Mole Fraction (ppb) & Emissions (nmol m\textsuperscript{-2} s\textsuperscript{-1}) \\
\midrule
Nitric oxide & NO & 0.1 & 31.8 \\
Nitrogen dioxide & NO\textsubscript{2} & 1.0 & 1.67 \\
Nitric acid & HNO\textsubscript{3} & 1.0 & \\
Ozone & O\textsubscript{3} & 50.0 & \\
Hydrogen peroxide & H\textsubscript{2}O\textsubscript{2} & 1.1 & \\
Carbon monoxide & CO & 21 & 291.3 \\
Sulfur dioxide & SO\textsubscript{2} & 0.8 & 2.51 \\
Ammonia & NH\textsubscript{3} & 0.5 & 6.11 \\
Hydrogen chloride & HCl & 0.7 & \\
Methane & CH\textsubscript{4} & 2200 & \\
Ethane & C\textsubscript{2}H\textsubscript{6} & 1.0 & \\
Formaldehyde & HCHO & 1.2 & 1.68 \\
Methanol & CH\textsubscript{3}OH & 0.12 & 0.28 \\
Methyl hydrogen peroxide & CH\textsubscript{3}OOH & 0.5 & \\
Acetaldehyde & ALD2 & 1.0 & 0.68 \\
Paraffin carbon & PAR & 2.0 & 96 \\
Acetone & AONE & 1.0 & 1.23 \\
Ethene & ETH & 0.2 & 7.2 \\
Terminal olefin carbons & OLET & 2.3 \(\cdot 10^{-2}\) & 2.42 \\
Internal olefin carbons & OLEI & 3.1 \(\cdot 10^{-4}\) & 2.42 \\
Toluene & TOL & 0.1 & 4.04 \\
Xylene & XYL & 0.1 & 2.41 \\
Lumped organic nitrate & ONIT & 0.1 & \\
Peroxyacetyl nitrate & PAN & 0.8 & \\
Higher organic acid & RCOOH & 0.2 & \\
Higher organic peroxide & ROOH & 2.5 \(\cdot 10^{-2}\) & \\
Isoprene & ISOP & 0.5 & 0.23 \\
Alcohols & ANOL & & 3.45 \\
\\[-2ex]\hline 
     \hline \\[-2ex]
\end{tabular*}
\label{table:gas_emiss_ics}
\end{table}


\begin{table}[!t]
\centering
\caption{Aerosol emissions and initial conditions. Table adapted from \citet{riemer_simulating_2009} with permission.}
\begin{tabular*}{\linewidth}{@{\extracolsep{\fill}} cccccc}
\\[-2ex]\hline 
     \hline \\[-2ex] Initial/Background  & $N$ (m$^{-3}$) & $D_{\text{gn}}$ ($\mu$m) & $\sigma_g$ & Composition by Mass\\
 \midrule
Aitken Mode & $3.2 \cdot 10^9$ & 0.02 & 1.45 & 50\% (NH$_4$)$_2$SO$_4$, 50\% POA\\
Accumulation Mode & $2.9 \cdot 10^9$ & 0.116 & 1.65 & 50\% (NH$_4$)$_2$SO$_4$, 50\% POA\\
\midrule
Emissions & $E$ (m$^{-2}$ s$^{-1}$) & $D_{\text{gn}}$ ($\mu$m) & $\sigma_g$ & Composition by Mass\\
\midrule
Meat cooking & $9 \cdot 10^6$ & 0.086 & 1.9 & 100\% POA\\
Diesel vehicles & $1.6 \cdot 10^8$ & 0.05 & 1.7 & 30\% POA, 70\% BC \\
Gasoline vehicles & $5 \cdot 10^7$ & 0.05 & 1.7 & 80\% POA, 20\% BC \\
\\[-2ex]\hline 
     \hline \\[-2ex]
\end{tabular*}
\label{table:aero_emiss_ics}
\end{table}


Meteorological initial conditions are specified using an idealized sounding for a convective boundary layer where the surface is 5 K warmer than the mixing layer and is capped by an inversion of 8 K at 1 km. %\hl{Maybe include a figure here?}
The wind profile is zero throughout the entire vertical extent of the domain.

%\hl{putting some namelist parameters here for time being for convenience}
%\begin{lstlisting}
% &physics
% mp_physics                          = 0,     0,     0, ! microphysics scheme
% ra_lw_physics                       = 0,     0,     0, ! long wave radiation scheme
% ra_sw_physics                       = 0,     0,     0, ! short wave radiation scheme
% radt                                = 0,     0,     0, ! radiation time step
% sf_sfclay_physics                   = 0,     1,     1, ! surface layer parameterization
% sf_surface_physics                  = 0,     0,     0, ! surface sub model
% bl_pbl_physics                      = 0,     0,     0, ! boundary layer parameterizations
% bldt                                = 0,     0,     0, !  boundary layer scheme time delta
% cu_physics                          = 0,     0,     0, ! cumulus physics model
% cudt                                = 0,     0,     0, ! cumulus physics scheme time delta
% isfflx                              = 2, ! surface heat and moisture flux
% num_land_cat = 24,
% num_soil_layers                     = 5, ! does this matter since I don't use a surface model?
%\end{lstlisting}


\subsection{Emissions Scenarios}

\begin{figure}[!t]
	\centering
	\includegraphics[]{figures/SH-scenarios.pdf}
	\caption{}
	\label{fig:sh-scenarios}
\end{figure} 

We evaluate the impacts of emissions spatial heterogeneity on aerosol properties via numerous emissions scenarios shown in Figure \ref{fig:sh-scenarios}. Emissions are first released uniformly over the entire domain. This scenario, referred to as the uniform base case, serves as a proxy for coarser resolution models which do not resolve the spatial heterogeneity of emissions and instead emit gas compounds and primary aerosol in a uniform and diffuse manner across grid cells. Successive scenarios are compared against this uniform base case to evaluate modifications to aerosol properties resulting from the spatial heterogeneity of emissions. Scenario 1 represents an idealized depiction of an urban-rural interface, whereby emissions are released in half the domain with no emissions occurring in the other half. Scenario 2 contains a narrow strip of emissions running through the center of the domain, corresponding to an emission pattern typical of a busy roadway. Lastly, scenario 3 places all emissions in a single grid cell in the center of the domain and is representative of a point-source emission, such as a plume from an industrial source.  

Emissions scenarios are chosen to represent a range of spatial heterogeneity, which is quantified using the SH metric of \citep{mohebalhojeh_2024} \hl{make sure to update this citation when Matin's paper is published}. The SH metric is a normalized measure of spatial heterogeneity, ranging from 0 (fully homogeneous) to 1 (fully heterogeneous). Thus, the uniform base case corresponds to the homogeneous condition while scenario 3---point source emission---represents the maximally heterogeneous emissions scenario. To ensure the mass of gas species and aerosol emitted per unit of time is consistent across each scenario, emission rates are divided by the fraction of area covered by emissions. For instance, in scenario 3, this results in a scaling of $10,000$ for the point-source emission. The fraction of area covered by emissions is displayed for each scenario alongside the spatial heterogeneity as quantified via the metric of \citet{mohebalhojeh_2024} in Figure \ref{fig:sh-scenarios}.

\section{Results}

\subsection{Aerosol size distributions}

\begin{figure}[!h]
	\centering
	\includegraphics[]{figures/combined_num_mass_conc_i50_j50_k60.pdf}
	\caption{}
	\label{fig:size-dists}
\end{figure} 

Number and mass distributions for each emissions scenario are shown in Figure \ref{fig:size-dists}. The initial condition is corresponds to the black dashed line, indicating a bimodal distribution of Aitken and accumulation mode particles. Solid lines indicate the distribution at the end of each simulation ($t=6$ h). The uniform base case corresponds to the solid black line, where emissions scenarios are indicated by colored solid lines.  Each size distribution is taken from a vertical level in the upper boundary layer at $z\approx800$ m. Due to the stochastic treatment of aerosol particles in WRF-PartMC and the selected number of computational particles per grid cell ($N = 100$), both number and mass distributions represent the average distribution in a 1 km$^2$ region centered over the emissions plume (i.e., size distributions are averaged over a $10\times10$ grid cell region). For the uniform base case, scenario 2 and scenario 3, this region is directly over the center of the domain. For scenario 1, emissions are released in one half of the domain that is offset from the center, and thus the averaging region is located in the center of the emissions patch. For each size distribution, data have been binned into 100 bins, ranging in size from $10^{-9}$ to $10^{-3}$ m. 

As the spatial heterogeneity of emissions increases from the uniform base case to scenario 3, we find that the number of Aitken mode particles decreases while the number of accumulation mode particles increases. This points to enhanced Brownian coagulation among ultra-fine particles as a result of higher local concentrations near the emissions plume core. 

For scenarios with high emissions spatial heterogeneity, the mass distribution in the accumulation mode increases while a slight decrease is observed in the Aitken mode. Coagulation of smaller Aitken mode particles with accumulation mode particles contributes little change in the mass distribution as indicated by a slight reduction in the Aitken mode mass concentration. This indicates that the increase in mass concentration in the accumulation mode is largely due to gas-particle partitioning, which will be discussed in detail in the next section.

\subsection{Aerosol composition}

\begin{figure}[!h]
	\centering
	\includegraphics[]{figures/aerosol-SNA-vertical-profiles-time36.pdf}
	\caption{Modify the linewidth, a bit too thick}
	\label{fig:vertical-profile-SNA}
\end{figure} 

Figure \ref{fig:vertical-profile-SNA} shows vertical profiles of aerosol ammonium (NH$_4$), nitrate (NO$_3$), and sulfate (SO$_4$) for each emissions scenario. The uniform base case is indicated via the solid black line while scenarios are indicated by colored lines ranging from blue (low spatial heterogeneity) to light green (high spatial heterogeneity). These profiles represent the average concentration of aerosol species within each vertical level at the end of each simulation ($t=6$ h). 

Sulfate concentrations are nearly uniform within the boundary layer and rapidly decrease above the entrainment zone due to little exchange between the free troposphere and the boundary layer. Sulfate concentrations decrease as the emissions spatial heterogeneity is increased. Note that for scenarios with high emissions spatial heterogeneity, the elevated concentration of reactive gas phase compounds in the emissions plume such as volatile organic compounds (VOCs) alongside sulfate's gas phase precursor SO$_2$ results in greater effective competition for oxidation with OH. As a result, OH is rapidly depleted near the emissions plume, thereby reducing the potential for oxidation of SO$_2$ to form H$_2$SO$_4$ (which rapidly partitions into the aerosol phase as sulfate owing to its extremely low volatility vapor pressure). OH from outside the emissions plume is not mixed and entrained into the plume fast enough to restore its concentration, thereby illustrating the impact of emissions spatial heterogeneity in chemically segregating reactive gas phase species OH and SO$_2$, and in altering the subsequent formation of sulfate through gas-particle partitioning.

Both ammonium and nitrate concentrations increase with height in the boundary layer due to the strong temperature dependence of ammonium nitrate formation. The abundance of nitrate depends on the availability of free ammonia, that is, ammonia in excess of what is required to neutralize sulfate as ammonium sulfate. In the lowest 500~m of the boundary layer, the concentration of NH$_4$ decreases as the emissions spatial heterogeneity increases. \hl{What might be the physical reasoning for this? Perhaps the lower concentration of sulfate in high SH scenarios drives the equilibrium condition to the gas phase? Note that when the particles are sulfate rich, they are highly acidic and thus allow absorption of any available NH3 so perhaps the reverse is occurring in this case?}. As emissions spatial heterogeneity increases, greater 
NH$_4$ and NO$_3$ form as ammonium nitrate in the aerosol phase towards the top of the boundary layer ($z\sim1.2$ km at $t=6$ h). The decrease in sulfate concentrations for scenarios with high emissions spatial heterogeneity results in higher concentrations of free ammonia, thus allowing the formation of more ammonium nitrate. Note that in the uniform base case, almost no nitrate is formed due to the lack of free ammonium as nearly all NH$_4$ is bound to sulfate as ammonium sulfate. This indicates the strong dependence of nitrate concentrations on the composition of the aerosol and the level of emissions spatial heterogeneity. 

\begin{figure}[!h]
	\centering
	\includegraphics[]{figures/speciated-mass-frac-three-panel-z40.pdf}
	\caption{Size-resolved mass fraction for emissions scenario~3 at regular 2-hour intervals showing species mass fraction as a percent of total aerosol mass vs. particle diameter.}
	\label{fig:speciated-mass-frac}
\end{figure} 

Figure \ref{fig:speciated-mass-frac} shows the size-resolved mass fraction of aerosols in the upper boundary layer ($z\approx 800$ m) for the initial condition and at the end of simulations ($t=6$ h) for both the uniform base case and the maximally spatially heterogeneous emissions scenario, scenario 3. After 6 hours, a stark difference in composition is apparent between particles in the uniform base case and scenario 3. Under uniform, dilute emissions, particles are largely comprised of BC and organic carbon (OC) along with some sulfate. Particles that age under the spatially heterogeneous emissions of scenario 3 are dominated by nitrate, ammonium, and sulfate and jointly comprise between 50 to 80\% of aerosol mass. 

The CCN activity of particles in the size range of 50-100 nm is largely dependent on their composition---in the presence of hygroscopic aerosol, the solute effect of Raoult's Law lowers the critical supersaturation required for activation. Without the aid of hygroscopic material, small aerosol particles possess high critical supersaturations due to the Kelvin effect which increases vapor pressure over the particle surface due to its curvature. Figure \ref{fig:speciated-mass-frac} indicates that, for the uniform base case, particles in the size range of 50-100 nm are primarily composed of BC and OC (low hygroscopicity compounds) whereas sulfate, nitrate, and ammonium dominate the mass fraction in the same size range for the high emissions spatial heterogeneity scenario. This indicates that particles in the size range whose CCN activity hinges on aerosol composition are more hygroscopic under high emissions spatial heterogeneity scenarios and thus activate at lower supersaturations.

\begin{figure}[!t]
	\centering
	\includegraphics[]{figures/2d-kappa-dist-three-panel-z40.pdf}
	\caption{2-dimensional number distributions $n(D_p, \kappa)$ at regular two-hour intervals for scenario~3.}
	\label{fig:kappa-dist}
\end{figure} 

Figure \ref{fig:kappa-dist} shows 2-dimensional number distributions $n(D_p, \kappa)$ as a function of particle diameter $D_p$ and particle hygroscopicity parameter $\kappa$ in the upper boundary layer ($z\approx 800$~m). Particles are binned into a two-dimensional histogram by diameter (50 bins ranging from 10 nm to 1 $\mu$m) and bins are colored by their corresponding number concentration indicated by the colorbar. The initial condition is shown on the left whereby all particles possess the same composition and thus the same hygroscopicity. The inclusion of ammonium sulfate in the initial condition makes the particles moderately hygroscopic. On the right, the 2-dimensional size distribution is shown at the end of simulations ($t=6$~h) for both the uniform base case and the maximally spatially heterogeneous scenario, scenario 3. 

Two primary groupings of particles are noticeable for the uniform base case---a grouping of lower $\kappa$ particles between $\kappa=0$ and $\kappa\approx0.3$ alongside a grouping of particles with higher $\kappa$ ranging from $\kappa\approx0.3$ to $\kappa\approx0.6$. The grouping of lower $\kappa$ particles correspond to carbonacous primary aerosol which have not undergone significant aging in the form of coagulation and gas-particle partitioning which increase the fraction of hygroscopic compounds (such as sulfate, nitrate, and ammonium). The upper band of higher $\kappa$ particles is shown to increase towards smaller particles, consistent with the elevated levels of sulfate observed in particles between 20--50 nm in the uniform base case shown in Figure \ref{fig:speciated-mass-frac}.

By comparison to the uniform base case, the number distribution for scenario 3 indicates the presence of higher $\kappa$ particles. For instance, the hygroscopicity of particles with diameter of 100 nm is in excess of $\kappa>0.6$ for scenario 3 (indicating highly hygroscopic particles), whereas $\kappa$ only reaches up to 0.4 for 100 nm particles in the uniform base case. These findings show that spatially heterogeneous emissions can elevate the hygroscopicity of particles whose CCN activity is dependent on particle composition ($D_p\sim50\text{--}100$ nm).  

Note that differences in the 2-dimensional number distributions with regard to $\kappa$ between the uniform base case and scenario 3 can be attributed to the coupling between emissions spatial heterogeneity and sub-grid scale aerosol processes. As previously noted, the enhancement of coagulation in spatially heterogeneous emission plumes reduces the concentration of Aitken mode particles. This helps explain the absence of a grouping of low-$\kappa$ carbonaceous primary aerosol particles. Furthermore, enhancements to gas-particle partitioning due to spatially heterogeneous emissions are responsible for increasing the hygroscopicity of particles. In particular, the formation of ammonium nitrate in scenario 3 due to reduced levels of sulfate and a corresponding increase in free ammonia elevate particle hygroscopicity. 


\subsection{CCN activity}

\begin{figure}[!h]
	\centering
	\includegraphics[]{figures/aerosol-ccn-vertical-profiles-time36.pdf}
	\caption{}
	\label{fig:ccn-vertical-prof}
\end{figure} 

Figure \ref{fig:ccn-vertical-prof} shows vertical profiles of the CCN number concentration per kilogram of dry air for supersaturations $S$ ranging from $S=0.1\%$ to $S=1.0\%$ and for each emissions scenario. Note that the ambient relative humidity (RH) in each simulation does not exceed 100\%---instead, reported CCN concentrations indicate the number of particles that would activate given the RH were raised to the specified supersaturation.

As noted previously, the coupling between emissions spatial heterogeneity and aerosol processes such as coagulation and gas-particle partitioning are responsible for altering the number, size, composition, and hygroscopicity of particles. These processes in turn modify the resulting CCN activity; however, the dominant processes contributing to changes in CCN activity and the strength of effect vary with supersaturation. 

At low supersaturations ($S=0.1\text{--}0.3\%$), CCN activity increases with emissions spatial heterogeneity in the upper boundary layer due to enhanced formation of ammonium nitrate in the cooler, sulfate poor environment. This elevates activation of ultrafine particles in the range of 50--100 nm due to the high hygroscopicity of ammonium nitrate. 

At higher supersaturations ($S=0.6\text{--}1.0\%$), an increase in CCN activity in the upper boundary layer is still found for emissions scenarios with lower spatial heterogeneity, however, at high emissions heterogeneity (scenario 3), the number concentration is reduced. This effect is most prominent at the highest supersaturation ($S=1.0\%$), whereby the number concentration of CCN throughout the boundary layer is less than all other scenarios including the uniform base case. This phenomenon is due to the enhancement of coagulation by the emissions spatial heterogeneity and illustrates how coagulation has a competing effect on CCN activity which dominates at high supersaturations. Note that small, non-hygroscopic particles possess high critical supersaturations. Additionally, the enhancement to coagulation for emissions scenarios with high spatial heterogeneity acts to efficiently remove smaller particles. Therefore, at sufficiently high supersaturation and emissions spatial heterogeneity, the negative effect on CCN activity due to coagulation will dominate the positive effect introduced by gas-particle partitioning of hygroscopic material. 

\begin{figure}[!h]
	\centering
	\includegraphics[]{figures/height-time-ccn-pdiff-multi-scenario.pdf}
	\caption{}
	\label{fig:time-height-ccn-pdiff}
\end{figure} 

Figure \ref{fig:time-height-ccn-pdiff} shows time vs. height plots of the percent difference between the CCN concentrations at each supersaturation level and emissions scenario and the corresponding CCN concentrations in the uniform base case. Percent difference is calculated as 
\begin{equation}
    \% \text{ difference} = 100\times\left(\frac{\overline{[\text{CCN}]}(t, z, S)_{\text{Scenario}} - \overline{[\text{CCN}]}(t, z, S)_{\text{Base case}}}{\overline{[\text{CCN}]}(t, z, S)_{\text{Base case}}}\right),
\end{equation}
where $\overline{[\text{CCN}]}(t, z,S)_{\text{Scenario}}$ is the horizontally averaged concentration of CCN at time $t$ and vertical level $z$ that activate at supersaturation $S$ for the given emissions scenario and $\overline{[\text{CCN}]}(t, z, S)_{\text{Base case}}$ is the horizontally averaged concentration of CCN at time $t$ and vertical level $z$ that activate at supersaturation $S$ for the uniform base case. 

The greatest increase in CCN activity occurs for scenario 3, the highest spatial heterogeneity emissions scenario, at a supersaturation of $S=0.3\%$ whereby CCN concentrations increase by upwards of 25\% through $t=6$ h. Across each scenario, the location of greatest increase in CCN activity grows with time due to boundary layer development. Moving down (higher supersaturation) and to the right (higher emissions spatial heterogeneity) in Figure \ref{fig:time-height-ccn-pdiff}, the reduction in CCN activity due to the enhancement of coagulation (see discussion of Figure \ref{fig:ccn-vertical-prof} for a process-level description) manifests after approximately 5 hours of simulation. \hl{Anything else to say about this figure?}

%\begin{figure}[!t]
%	\centering
%	\includegraphics[]{figures/4panel-ccn-rel-err-vs-sh.pdf}
%	\caption{Make absolute value plots and remove coloring by height20}
%	%\label{fig:transport-vs-aerosol-model}
%\end{figure} 


\subsection{Influence of ammonia on aerosol composition and CCN activity}

We have shown that CCN activity is modulated by emissions spatial heterogeneity due to enhancements to coagulation and the gas-particle partitioning of ammonia and nitric acid to form ammonium nitrate which elevates particle hygroscopicity. Furthermore, the formation of ammonium nitrate is governed by the availability of free ammonia. To investigate the sensitivity of changes in CCN activity due to the presence of ammonia in spatially heterogeneous emissions, a set of two additional simulations were run for the uniform base case and scenario 3 in which the concentration of total ammonium (NH$_{3\text{, gas}} + $NH$_{4\text{, aerosol}}$) is set to zero. Emissions of NH$_3$ are also set to zero to ensure that total ammonium remains zero throughout each simulation. 

\begin{figure}[!h]
	\centering
	\includegraphics[]{figures/aerosol-ccn-vertical-profiles-no-nh4-cases-time36.pdf}
	\caption{}
	\label{fig:ccn-vertical-profile-no-ammonia}
\end{figure}

Figure \ref{fig:ccn-vertical-profile-no-ammonia} shows vertical profiles of CCN concentrations in number of particles per kilogram of dry air at supersaturations ranging from $S=0.1\%$ to $S=1.0\%$ and at $t=6$ h. Profiles for simulations containing ammonia are displayed as solid lines while simulations without ammonia are shown as dashed lines. We find that, without ammonia, CCN concentrations at each supersaturation level agree much closer. The peak of CCN concentrations in the upper boundary layer and at lower supersaturations found for scenario 3 with ammonia is entirely absent. This underscores the significance of the coupling between spatially heterogeneous emissions, their composition, and the formation of ammonium nitrate via gas-particle partitioning in altering CCN activity, especially at lower supersaturations. 

At higher supersaturations, particularly at $S=1.0\%$, the reduction in CCN concentrations for scenario 3 without ammonia relative to the corresponding uniform base case point to the enhancement of coagulation under spatially heterogeneous emissions. Without the countervailing effect of gas-particle partitioning in increasing CCN activity due to ammonium nitrate formation, the vertical profiles of the ammonia-free uniform base case and scenario 3 closely match each other, albeit with an offset for scenario 3 due to the enhancement of coagulation that shifts the number concentration lower by approximately 5\%.  

\conclusions  %% \conclusions[modified heading if necessary]
This study investigates the impacts of spatially heterogeneous emissions on aerosol properties in a convective boundary layer including CCN activity using a first-of-a-kind particle-resolved large-eddy simulation modeling framework, WRF-PartMC-MOSAIC-LES. This platform permits a process level analysis of the coupling between emissions spatial heterogeneity and concentration dependent aerosol processes such as coagulation and gas-particle partitioning. Emissions spatial heterogeneity is varied using numerous idealized scenarios and are compared against a base case of uniform emissions which acts as a proxy for coarser-resolved models whose grid resolution is insufficient to represent the underlying spatial heterogeneity of emissions.

Key aerosol processes including gas-particle partitioning and coagulation are impacted by emissions spatial heterogeneity, as we find significant changes to the sulfate-nitrate-ammonium system and an increased rate of coagulation under high emissions spatial heterogeneity scenarios.

Changes to aerosol processes have downstream effects on CCN activity. Furthermore, modifications by emissions spatial heterogeneity to coagulation and gas-particle partitioning result in competing effects on the concentration of CCN at a given supersaturation level. Coagulation removes smaller particles that activate at high supersaturations, resulting in a decrease in CCN activity at high supersaturations for scenarios with high emissions spatial heterogeneity. Conversely, coagulation is not as efficient at removing larger particles that activate at lower supersaturations and gas-particle partitioning results in an increase of highly hygroscopic compounds such as ammonium nitrate under high emissions spatial heterogeneity. As a result, CCN activity at lower supersaturations ($S = 0.3\mbox{–-}0.6\%$) increases by up to 25\% in the upper boundary layer for emissions scenarios with high spatial heterogeneity.

The effect of emissions spatial heterogeneity on CCN activity is highly dependent on the composition of the aerosol and gas phase. Given the key contribution of ammonium nitrate formation in elevating CCN activity under highly spatially heterogeneous scenarios, the removal of ammonia decreases---or in some cases reverses---the trend between emissions spatial heterogeneity and CCN activity.  

%\begin{itemize}
%\item Limitations and future work
%\end{itemize}

%% The following commands are for the statements about the availability of data sets and/or software code corresponding to the manuscript.
%% It is strongly recommended to make use of these sections in case data sets and/or software code have been part of your research the article is based on.

\codeavailability{TEXT} %% use this section when having only software code available


\dataavailability{TEXT} %% use this section when having only data sets available


\codedataavailability{TEXT} %% use this section when having data sets and software code available


\sampleavailability{TEXT} %% use this section when having geoscientific samples available


\videosupplement{TEXT} %% use this section when having video supplements available


\appendix
\section{}    %% Appendix A

\subsection{}     %% Appendix A1, A2, etc.


\noappendix       %% use this to mark the end of the appendix section. Otherwise the figures might be numbered incorrectly (e.g. 10 instead of 1).

%% Regarding figures and tables in appendices, the following two options are possible depending on your general handling of figures and tables in the manuscript environment:

%% Option 1: If you sorted all figures and tables into the sections of the text, please also sort the appendix figures and appendix tables into the respective appendix sections.
%% They will be correctly named automatically.

%% Option 2: If you put all figures after the reference list, please insert appendix tables and figures after the normal tables and figures.
%% To rename them correctly to A1, A2, etc., please add the following commands in front of them:

\appendixfigures  %% needs to be added in front of appendix figures

\appendixtables   %% needs to be added in front of appendix tables

%% Please add \clearpage between each table and/or figure. Further guidelines on figures and tables can be found below.



\authorcontribution{TEXT} %% this section is mandatory

\competinginterests{TEXT} %% this section is mandatory even if you declare that no competing interests are present

\disclaimer{TEXT} %% optional section

\begin{acknowledgements}
TEXT
\end{acknowledgements}




%% REFERENCES

%% The reference list is compiled as follows:

%\begin{thebibliography}{}

%\bibitem[AUTHOR(YEAR)]{LABEL1}
%REFERENCE 1

%\bibitem[AUTHOR(YEAR)]{LABEL2}
%REFERENCE 2

%\end{thebibliography}

%% Since the Copernicus LaTeX package includes the BibTeX style file copernicus.bst,
%% authors experienced with BibTeX only have to include the following two lines:
%%
\bibliographystyle{copernicus}
\bibliography{all-references.bib}
%%
%% URLs and DOIs can be entered in your BibTeX file as:
%%
%% URL = {http://www.xyz.org/~jones/idx_g.htm}
%% DOI = {10.5194/xyz}


%% LITERATURE CITATIONS
%%
%% command                        & example result
%% \citet{jones90}|               & Jones et al. (1990)
%% \citep{jones90}|               & (Jones et al., 1990)
%% \citep{jones90,jones93}|       & (Jones et al., 1990, 1993)
%% \citep[p.~32]{jones90}|        & (Jones et al., 1990, p.~32)
%% \citep[e.g.,][]{jones90}|      & (e.g., Jones et al., 1990)
%% \citep[e.g.,][p.~32]{jones90}| & (e.g., Jones et al., 1990, p.~32)
%% \citeauthor{jones90}|          & Jones et al.
%% \citeyear{jones90}|            & 1990



%% FIGURES

%% When figures and tables are placed at the end of the MS (article in one-column style), please add \clearpage
%% between bibliography and first table and/or figure as well as between each table and/or figure.

% The figure files should be labelled correctly with Arabic numerals (e.g. fig01.jpg, fig02.png).


%% ONE-COLUMN FIGURES

%%f
%\begin{figure}[t]
%\includegraphics[width=8.3cm]{FILE NAME}
%\caption{TEXT}
%\end{figure}
%
%%% TWO-COLUMN FIGURES
%
%%f
%\begin{figure*}[t]
%\includegraphics[width=12cm]{FILE NAME}
%\caption{TEXT}
%\end{figure*}
%
%
%%% TABLES
%%%
%%% The different columns must be seperated with a & command and should
%%% end with \\ to identify the column brake.
%
%%% ONE-COLUMN TABLE
%
%%t
%\begin{table}[t]
%\caption{TEXT}
%\begin{tabular}{column = lcr}
%\tophline
%
%\middlehline
%
%\bottomhline
%\end{tabular}
%\belowtable{} % Table Footnotes
%\end{table}
%
%%% TWO-COLUMN TABLE
%
%%t
%\begin{table*}[t]
%\caption{TEXT}
%\begin{tabular}{column = lcr}
%\tophline
%
%\middlehline
%
%\bottomhline
%\end{tabular}
%\belowtable{} % Table Footnotes
%\end{table*}
%
%%% LANDSCAPE TABLE
%
%%t
%\begin{sidewaystable*}[t]
%\caption{TEXT}
%\begin{tabular}{column = lcr}
%\tophline
%
%\middlehline
%
%\bottomhline
%\end{tabular}
%\belowtable{} % Table Footnotes
%\end{sidewaystable*}
%
%
%%% MATHEMATICAL EXPRESSIONS
%
%%% All papers typeset by Copernicus Publications follow the math typesetting regulations
%%% given by the IUPAC Green Book (IUPAC: Quantities, Units and Symbols in Physical Chemistry,
%%% 2nd Edn., Blackwell Science, available at: http://old.iupac.org/publications/books/gbook/green_book_2ed.pdf, 1993).
%%%
%%% Physical quantities/variables are typeset in italic font (t for time, T for Temperature)
%%% Indices which are not defined are typeset in italic font (x, y, z, a, b, c)
%%% Items/objects which are defined are typeset in roman font (Car A, Car B)
%%% Descriptions/specifications which are defined by itself are typeset in roman font (abs, rel, ref, tot, net, ice)
%%% Abbreviations from 2 letters are typeset in roman font (RH, LAI)
%%% Vectors are identified in bold italic font using \vec{x}
%%% Matrices are identified in bold roman font
%%% Multiplication signs are typeset using the LaTeX commands \times (for vector products, grids, and exponential notations) or \cdot
%%% The character * should not be applied as mutliplication sign
%
%
%%% EQUATIONS
%
%%% Single-row equation
%
%\begin{equation}
%
%\end{equation}
%
%%% Multiline equation
%
%\begin{align}
%& 3 + 5 = 8\\
%& 3 + 5 = 8\\
%& 3 + 5 = 8
%\end{align}
%
%
%%% MATRICES
%
%\begin{matrix}
%x & y & z\\
%x & y & z\\
%x & y & z\\
%\end{matrix}
%
%
%%% ALGORITHM
%
%\begin{algorithm}
%\caption{...}
%\label{a1}
%\begin{algorithmic}
%...
%\end{algorithmic}
%\end{algorithm}
%
%
%%% CHEMICAL FORMULAS AND REACTIONS
%
%%% For formulas embedded in the text, please use \chem{}
%
%%% The reaction environment creates labels including the letter R, i.e. (R1), (R2), etc.
%
%\begin{reaction}
%%% \rightarrow should be used for normal (one-way) chemical reactions
%%% \rightleftharpoons should be used for equilibria
%%% \leftrightarrow should be used for resonance structures
%\end{reaction}
%
%
%%% PHYSICAL UNITS
%%%
%%% Please use \unit{} and apply the exponential notation


\end{document}
