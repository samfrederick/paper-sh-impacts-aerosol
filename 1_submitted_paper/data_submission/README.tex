\documentclass[gmd,manuscript]{copernicus}
\usepackage{forest}
\usepackage{pdflscape}

\begin{document}

\title{Data for Idealized Particle-Resolved Large-Eddy Simulations to Evaluate the Impact of Emissions Spatial Heterogeneity on CCN Activity}

\Author[1]{Samuel}{Frederick}
\Author[2]{Matin}{Mohebalhojeh}
\Author[1,2]{Jeffrey}{Curtis}
\Author[2]{Matthew}{West}
\Author[1][nriemer@illinois.edu]{Nicole}{Riemer} %% correspondence author

\affil[1]{Department of Climate, Meteorology, and Atmospheric Sciences, University of Illinois Urbana--Champaign, 1301 W. Green St., Urbana, IL 61801, USA}
\affil[2]{Department of Mechanical Science and Engineering, University of Illinois Urbana--Champaign, 1206 W. Green St., Urbana, IL, 61801,USA}

\correspondence{Nicole Riemer (nriemer@illinois.edu)}

\runningtitle{Data for \textcolor{red}{Copy final name from paper}}

\runningauthor{Frederick et al.}

%% These dates will be inserted by Copernicus Publications during the typesetting process.


\firstpage{1}

\maketitle

This dataset contains all material required to produce the figures found within the manuscript submitted to Aerosol Chemistry and Physics entitled ``Idealized Particle-Resolved Large-Eddy Simulations to Evaluate the Impact of Emissions Spatial Heterogeneity on CCN Activity''. 
The archived dataset consists of:

\begin{itemize}
\item \texttt{data.zip}: WRF-PartMC-MOSIAC-LES simulation data in Section~3
\item \texttt{scripts.zip}: Python notebooks for generating figures in Section~2 and 3, and scripts used to compile the normalized spatial heterogeneity metric as described by Mohebalhojeh et al. 2025.
\end{itemize}

\section*{Software requirements/recommendations}

All figures in the paper were run with Python 3.9.23. This older version is required due to a dependency of \texttt{f2py} (used to compile Fortran code for the spatial heterogeneity metric calculation into an object which can be imported as a Python module) which has since been deprecated.  Users should run the \texttt{create\_env.sh} script in \texttt{scripts.zip} to create a conda environment which contains all the necessary packages.

Required packages are as follows with the version used for this manuscript:
\begin{itemize}
\item numpy (2.0.2)
\item scipy (1.13.1)
\item matplotlib (3.9.2)
\item netCDF4 (1.7.2) - available at \url{https://unidata.github.io/netcdf4-python/}
\item pandas (2.3.1)
\item ipykernel (6.30.1)
\item setuptools (59.8.0)
\item gfortran (15.1.0)
\end{itemize}


\section*{Directory structure of archived simulation data for Section~3.1}

Upon downloading and untarring \texttt{partmc\_simulations.tar.gz},  it may be explored as follows:

\begin{itemize}
    \item Input data files for conducting the set of three simulations:
    \begin{itemize}
      \item \texttt{camp.spec} and relevant input files
      \item \texttt{tchem.spec} and relevant input files
      \item \texttt{tchem\_gpu.spec} and relevant input files
    \end{itemize}
    \item Output data in the \texttt{out/} directory consists of netCDF files per output time:
	\begin{itemize}
	  \item CAMP output: \texttt{camp\_0001\_*}
	  \item PartMC-TChem CPU output: \texttt{tchem\_cb05cl\_ae5\_*}
	  \item PartMC-TChem GPU output: \texttt{tchem\_gpu\_cb05cl\_ae5\_*}
	\end{itemize}
	\item \texttt{simulation\_notebook.ipynb}: Python Jupyter notebook for producing Fig.~5 and analysis of error.
\end{itemize}

\section*{Directory structure of archived numerical experiment data  for Section~3.2}

Upon downloading and untarring \texttt{timings.tar.gz}, it may be explored as follows:

\begin{itemize}
   \item \texttt{data/}: directory containing all the timing results. The descriptions of the individual directories are described in Table~\ref{tab:experiment_configs}.
\item \texttt{scripts/}: directory containing the following scripts:
\begin{itemize}
   \item \texttt{solver\_plots.ipynb}: Python Jupyter notebook for producing Fig.~6, 7.
   \item \texttt{rhs\_plots.ipynb}: Python Jupyter notebook for producing Fig.~8.
\end{itemize}
\end{itemize}

%\begin{landscape}
\begin{table}[ht]
\centering
\footnotesize
\begin{tabular}{|l|l|l|p{9.5cm}|}
\hline
\textbf{Cluster} & \textbf{Experiment} & \textbf{Platform} & \textbf{Path relative to data directory} \\
\hline
DeltaAI & RHS & NVIDIA H100 GPU & deltaAI/CB05CL\_AE5\_w\_simpolSOA/CUDA/rhs-no\_sacado \\
DeltaAI & Jacobian & NVIDIA H100 GPU & deltaAI/CB05CL\_AE5\_w\_simpolSOA/CUDA/rhss-no\_sacado \\
DeltaAI & TrBDF2 & NVIDIA H100 GPU & deltaAI/CB05CL\_AE5\_w\_simpolSOA/CUDA/trbdf-no\_sacado \\
DeltaAI & Sundials CVODE & NVIDIA H100 GPU & deltaAI/CB05CL\_AE5\_w\_simpolSOA/CUDA/sundials\_dense-no\_sacado \\
DeltaAI & Sundials CVODE-GMRES & NVIDIA H100 GPU & deltaAI/CB05CL\_AE5\_w\_simpolSOA/CUDA/sundials\_gmres-no\_sacado \\
Frontier & RHS & AMD MI250X GPU & frontier/CB05CL\_AE5\_w\_simpolSOA/HIP/rhs-no\_sacado \\
Frontier & Jacobian & AMD MI250X GPU & frontier/CB05CL\_AE5\_w\_simpolSOA/HIP/jac-no\_sacado \\
Frontier & TrBDF2 & AMD MI250X GPU & frontier/CB05CL\_AE5\_w\_simpolSOA/HIP/trbdf-no\_sacado \\
Frontier & Sundials CVODE & AMD MI250X GPU & frontier/CB05CL\_AE5\_w\_simpolSOA/HIP/sundials\_dense-no\_sacado \\
Frontier & Sundials CVODE-GMRES & AMD MI250X GPU & frontier/CB05CL\_AE5\_w\_simpolSOA/HIP/sundials\_gmres-no\_sacado \\
Perlmutter & RHS & AMD 7763 CPU & perlmutter/CB05CL\_AE5\_w\_simpolSOA/HOST/rhs-no\_sacado \\
Perlmutter & Jacobian & AMD 7763 CPU & perlmutter/CB05CL\_AE5\_w\_simpolSOA/HOST/jac-no\_sacado \\
Perlmutter & TrBDF2 & AMD 7763 CPU & perlmutter/CB05CL\_AE5\_w\_simpolSOA/HOST/trbdf-no\_sacado \\
Perlmutter & Sundials CVODE & AMD 7763 CPU & perlmutter/CB05CL\_AE5\_w\_simpolSOA/HOST/sundials\_dense-no\_sacado \\
Perlmutter & Sundials CVODE-GMRES & AMD 7763 CPU & perlmutter/CB05CL\_AE5\_w\_simpolSOA/HOST/sundials\_gmres-no\_sacado \\
\hline
\end{tabular}
\caption{Experiment configurations across different clusters and platforms}
\label{tab:experiment_configs}
\end{table}
%\end{landscape}


\end{document}

\begin{forest}
  for tree={
    font=\ttfamily,
    grow'=0,
    child anchor=west,
    parent anchor=south,
    anchor=west,
    calign=first,
    edge path={
      \noexpand\path [draw, \forestoption{edge}]
      (!u.south west) +(7.5pt,0) |- node[fill,inner sep=1.25pt] {} (.child anchor)\forestoption{edge label};
    },
    before typesetting nodes={
      if n=1
        {insert before={[,phantom]}}
                {}
    },
    fit=band,
    before computing xy={l=15pt},
  }
[
]
\end{forest}
