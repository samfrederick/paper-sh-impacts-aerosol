\documentclass[gmd,manuscript]{copernicus}
\usepackage{forest}
\usepackage{pdflscape}

\begin{document}

\title{Data for Idealized Particle-Resolved Large-Eddy Simulations to Evaluate the Impact of Emissions Spatial Heterogeneity on CCN Activity}

\Author[1]{Samuel}{Frederick}
\Author[2]{Matin}{Mohebalhojeh}
\Author[1,2]{Jeffrey}{Curtis}
\Author[2]{Matthew}{West}
\Author[1][nriemer@illinois.edu]{Nicole}{Riemer} %% correspondence author

\affil[1]{Department of Climate, Meteorology, and Atmospheric Sciences, University of Illinois Urbana--Champaign, 1301 W. Green St., Urbana, IL 61801, USA}
\affil[2]{Department of Mechanical Science and Engineering, University of Illinois Urbana--Champaign, 1206 W. Green St., Urbana, IL, 61801,USA}

\correspondence{Nicole Riemer (nriemer@illinois.edu)}

\runningtitle{Data for \textcolor{red}{Copy final name from paper}}

\runningauthor{Frederick et al.}

%% These dates will be inserted by Copernicus Publications during the typesetting process.


\firstpage{1}

\maketitle

This dataset contains all material required to produce the figures found within the manuscript submitted to Aerosol Chemistry and Physics entitled ``Idealized Particle-Resolved Large-Eddy Simulations to Evaluate the Impact of Emissions Spatial Heterogeneity on CCN Activity''. 
The archived dataset consists of:

\begin{itemize}
\item \texttt{data.zip}: WRF-PartMC-MOSIAC-LES simulation data in Section~3
\item \texttt{scripts.zip}: Python notebooks for generating figures in Section~2 and 3, and scripts used to compile the normalized spatial heterogeneity metric as described by Mohebalhojeh et al. 2025.
\end{itemize}

\section*{Software requirements/recommendations}

All figures in the paper were run with Python 3.9.23. This older version is required due to a dependency of \texttt{f2py} (used to compile Fortran code for the spatial heterogeneity metric calculation into an object which can be imported as a Python module) which has since been deprecated.  Users should run the \texttt{create\_env.sh} script in \texttt{scripts.zip} to create a conda environment which contains all the necessary packages.

Required packages are as follows with the version used for this manuscript:
\begin{itemize}
\item numpy (2.0.2)
\item scipy (1.13.1)
\item matplotlib (3.9.2)
\item netCDF4 (1.7.2) - available at \url{https://unidata.github.io/netcdf4-python/}
\item pandas (2.3.1)
\item ipykernel (6.30.1)
\item setuptools (59.8.0)
\item gfortran (15.1.0)
\end{itemize}


\section*{Directory structure of simulation data for Section 3}

Upon downloading and untarring \texttt{data.zip}, it may be explored as follows:

\begin{itemize}
    \item Data files containing a subset of variables, spatial dimensions, and time slices used in generating Section 3 figures. Slices 
    of the domain indicate the index along a given dimension (e.g., time ranges from 0 to 36, vertical height ranges from 0 to 199). Datasets 
    are organized by each emissions scenario (\texttt{no-heterogeneity}, \texttt{low-heterogeneity}, \texttt{medium-heterogeneity}, \texttt{high-heterogeneity}). 
    Rather than listing each file here, we replace the scenario title in filenames by \texttt{*}:
    \begin{itemize}
      \item Per-grid cell gas mixing ratios, particle species mass fractions: \texttt{*\_subset\_t36.nc}
      \item Binned number and mass distributions: \texttt{*\_size-dist\_subset\_t0\_t36\_z60.nc}
      \item Detailed per-particle output (e.g., per-particle mass fractions, kappa): \texttt{crosssec\_*\_t36\_z40.nc}
      \item CCN concentration variables, averaged across each vertical level: \texttt{*\_ccn-vars\_subset.nc}
      \item Output for ammonia-free scenarios: \texttt{*-no-nh4\_subset\_t36.nc}
    \end{itemize}
    \item Data in the \texttt{spatial-het/} directory consists of lookup tables for spatial heterogeneity values and binary
    arrays indicating the structure of each emissions pattern:
	\begin{itemize}
	  \item Lookup table for the spatial heterogeneity value ($\eta$) of each emission pattern: \\
    \texttt{sh\_patterns\_xres100\_yres100\_exact.csv}
	  \item Arrays for each emission scenario: \texttt{/sh-patterns/xres100yres100/[scenario-name].csv}
	\end{itemize}
\end{itemize}

\section*{Directory structure of scripts for Section 3}

Upon downloading and untarring \texttt{scripts.zip}, it may be explored as follows:

\begin{itemize}
  \item \texttt{create\_env.sh}: Shell script for setting up a \texttt{conda} environment with the particular version of Python and associated packages required. \textbf{Please run this script first before proceededing to run other files in this directory.}
  \item \texttt{compile\_nsh.sh}: Shell script for compiling \texttt{nsh.f90} to a Python executable object via the \texttt{f2py} package. Once this script runs, you should see a \text{*.so} shared object library file.
  \item \texttt{nsh.f90}: Fortran module for calculating the discrete normalized spatial heterogeneity metric of Mohebalhojeh et al. 2025. This module contains two subroutines, \texttt{normalizedSpatialHet()} which is naive looping routine over all subarray configurations. For large domain sizes, this routine is computationally prohibitive and the Monte Carlo sampling subroutine \texttt{monteCarloSpatialHet()} is preferred.
  \item \texttt{griddedoutput\_helperfuncs.py}: Helper functions for processing and plotting per-particle datasets.
  \item \texttt{griddedoutput\_plotting.py}: Plotting functions for per-particle datasets.
  \item \texttt{loaddatastructs.py}: Datasets and associated attributes are housed within the objects \texttt{DataStruct} and the inherited class \texttt{GriddedOutput} for per-particle datasets.
  \item \texttt{paper-figures-bulk.ipynb}: Python notebook for generating Figures 1-6 and 9-11.  
  \item \texttt{paper-figures-particle-resolved.ipynb}: Python notebook for generating Figures 7,8 from per-particle datasets. 
\end{itemize}

\end{document}