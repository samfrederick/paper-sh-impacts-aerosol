%% Copernicus Publications Manuscript Preparation Template for LaTeX Submissions
%% ---------------------------------
%% This template should be used for copernicus.cls
%% The class file and some style files are bundled in the Copernicus Latex Package, which can be downloaded from the different journal webpages.
%% For further assistance please contact Copernicus Publications at: production@copernicus.org
%% https://publications.copernicus.org/for_authors/manuscript_preparation.html


%% Please use the following documentclass and journal abbreviations for preprints and final revised papers.

%% 2-column papers and preprints
\documentclass[journal abbreviation, manuscript]{copernicus}
%\documentclass[acp]{copernicus} % for some reason you have to run this a few times to get the abstract to appear


%% Journal abbreviations (please use the same for preprints and final revised papers)


% Advances in Geosciences (adgeo)
% Advances in Radio Science (ars)
% Advances in Science and Research (asr)
% Advances in Statistical Climatology, Meteorology and Oceanography (ascmo)
% Aerosol Research (ar)
% Annales Geophysicae (angeo)
% Archives Animal Breeding (aab)
% Atmospheric Chemistry and Physics (acp)
% Atmospheric Measurement Techniques (amt)
% Biogeosciences (bg)
% Climate of the Past (cp)
% DEUQUA Special Publications (deuquasp)
% Earth Surface Dynamics (esurf)
% Earth System Dynamics (esd)
% Earth System Science Data (essd)
% E&G Quaternary Science Journal (egqsj)
% EGUsphere (egusphere) | This is only for EGUsphere preprints submitted without relation to an EGU journal.
% European Journal of Mineralogy (ejm)
% Fossil Record (fr)
% Geochronology (gchron)
% Geographica Helvetica (gh)
% Geoscience Communication (gc)
% Geoscientific Instrumentation, Methods and Data Systems (gi)
% Geoscientific Model Development (gmd)
% History of Geo- and Space Sciences (hgss)
% Hydrology and Earth System Sciences (hess)
% Journal of Bone and Joint Infection (jbji)
% Journal of Micropalaeontology (jm)
% Journal of Sensors and Sensor Systems (jsss)
% Magnetic Resonance (mr)
% Mechanical Sciences (ms)
% Natural Hazards and Earth System Sciences (nhess)
% Nonlinear Processes in Geophysics (npg)
% Ocean Science (os)
% Polarforschung - Journal of the German Society for Polar Research (polf)
% Primate Biology (pb)
% Proceedings of the International Association of Hydrological Sciences (piahs)
% Safety of Nuclear Waste Disposal (sand)
% Scientific Drilling (sd)
% SOIL (soil)
% Solid Earth (se)
% State of the Planet (sp)
% The Cryosphere (tc)
% Weather and Climate Dynamics (wcd)
% Web Ecology (we)
% Wind Energy Science (wes)


%% \usepackage commands included in the copernicus.cls:
%\usepackage[german, english]{babel}
%\usepackage{tabularx}
%\usepackage{cancel}
%\usepackage{multirow}
%\usepackage{supertabular}
%\usepackage{algorithmic}
%\usepackage{algorithm}
%\usepackage{amsthm}
%\usepackage{float}
%\usepackage{subfig}
%\usepackage{rotating}
\usepackage{xcolor}
\usepackage{soul}
\usepackage{listings}
\usepackage{booktabs}


\begin{document}

\title{Idealized Particle-Resolved Large-Eddy Simulations to Evaluate the Impact of Emissions Spatial Heterogeneity on CCN Activity}

\Author[1]{Samuel}{Frederick}
\Author[2]{Matin}{Mohebalhojeh}
\Author[1,2]{Jeffrey}{Curtis}
\Author[2]{Matthew}{West}
\Author[1][nriemer@illinois.edu]{Nicole}{Riemer} %% correspondence author


\affil[1]{Department of Climate, Meteorology, and Atmospheric Sciences, University of Illinois Urbana--Champaign, 1301 W. Green St., Urbana, IL 61801, USA}
\affil[2]{Department of Mechanical Science and Engineering, University of Illinois Urbana--Champaign, 1206 W. Green St., Urbana, IL, 61801,USA}

%% If authors contributed equally, please mark the respective author names with an asterisk, e.g. "\Author[2,*]{Anton}{Smith}" and "\Author[3,*]{Bradley}{Miller}" and add a further affiliation: "\affil[*]{These authors contributed equally to this work.}".

\runningtitle{Impacts of Emissions SH on CCN activity}

\runningauthor{Frederick}

\received{}
\pubdiscuss{} %% only important for two-stage journals
\revised{}
\accepted{}
\published{}

%% These dates will be inserted by Copernicus Publications during the typesetting process.

\firstpage{1}

\maketitle

\begin{abstract}
Aerosol-cloud interactions remain a large source of uncertainty in global climate models (GCMs) due to complex, nonlinear processes that alter aerosol properties and the inability to represent the full compositional complexity of aerosol populations within large-scale modeling frameworks. The spatial resolution of GCMs is often coarser than the scale of the spatially varying emissions in the modeled geographic region. This results in diffuse, uniform concentration fields of primary aerosol and gas-phase species instead of spatially heterogeneous concentrations. Aerosol processes such as gas-particle partitioning and coagulation are concentration-dependent in a non-linear manner, and thus the representation of spatially heterogeneous emissions impacts aerosol aging and properties. This includes climate-relevant quantities key to aerosol-cloud interactions including particle hygroscopicity and cloud condensation nuclei (CCN) activity. We investigate the impact of emissions spatial heterogeneity on aerosol properties including CCN activity via a series of first-of-a-kind particle-resolved large-eddy simulations with the modeling framework WRF-PartMC-MOSAIC-LES. CCN concentrations within the planetary boundary layer (PBL) are compared across numerous scenarios ranging in emissions spatial heterogeneity. We find that CCN concentrations at low supersaturations ($S=0.1\mbox{--}0.3\%$) increase in the upper PBL by up to 25\%
for emissions scenarios with high spatial heterogeneity when compared to a uniform emissions base case. Process level analysis indicates that this increase is due to enhanced nitrate formation among scenarios with high emissions spatial heterogeneity. 
\end{abstract}


%\copyrightstatement{TEXT} %% This section is optional and can be used for copyright transfers.


\introduction  %% \introduction[modified heading if necessary]

Aerosols exert a net negative radiative forcing, but significant
uncertainty remains in how aerosol-cloud interactions are
represented in climate models \citep{ipcc_report_2021}. Two major
factors contribute to this uncertainty: (1) the spatial resolution of
models and (2) the treatment of aerosol representation. Advances in
computational power have allowed modelers to investigate how spatial
resolution affects radiative forcing due to aerosol-cloud interactions
\citep{ma_how_2015}. At the same time, increasing computational
capabilities have enabled more sophisticated aerosol representations
\citep{zaveri_development_2021, tilmes_description_2023}. While both
spatial resolution and aerosol representation have advanced, they have
largely been studied in isolation. The combined effect of
sub-grid-scale spatial heterogeneity and detailed aerosol
representation on relevant properties, such as cloud
condensation nuclei (CCN) activity, remains largely unexplored.

Aerosol-aware climate models typically include a sub-model that
governs aerosol representation and associated processes. Due to
computational constraints, these aerosol treatments often simplify
compositional diversity. For instance, the Energy Exascale Earth
System Model (E3SM) uses the MAM4 scheme, which represents aerosols
using four internally mixed, lognormally distributed modes
\citep{golaz_doe_2022}. This approach inherently constrains the
diversity of aerosol populations, as all particles within a given mode
share identical composition. In reality, however, aerosols age
independently, leading to complex, highly heterogeneous
mixtures. Particle-resolved aerosol models address this limitation by
explicitly tracking the composition and evolution of individual
particles. The Particle Monte Carlo model (PartMC) has been
extensively used to study the sensitivity of CCN activity to aerosol
composition \citep{fierce_when_2013}, aging timescales due to
condensation and coagulation for carbonaceous CCN
\citep{fierce_explaining_2015}, and the impact of mixing state on CCN
estimates \citep{ching_metrics_2017}. Furthermore, PartMC has been
leveraged to quantify errors in CCN predictions from modal and
sectional models by comparing CCN concentrations against those derived
from fully resolved aerosol compositions
\citep{zaveri_particle-resolved_2010, ching_metrics_2017}. Comparisons
between PartMC and MAM4 have revealed substantial discrepancies in CCN
activity, particularly in polluted regions where coagulation and
gas-particle partitioning amplify model differences
\citep{fierce_quantifying_2024}.

Despite these advancements in aerosol representation, a critical gap
remains: the combined effects of high-resolution aerosol treatment and
fine-scale spatial heterogeneity have yet to be systematically
explored. Understanding this interaction is essential, as the
variability in surface properties, emissions, and resulting aerosol
plumes influences particle aging and CCN activity. The goal of this
study is to bridge this gap by conducting the first particle-resolved
large-eddy simulations (LES) to analyze how sub-grid-scale spatial
heterogeneity affects aerosol composition, aging, and CCN activation.

Existing regional and global-scale aerosol-aware models lack the
resolution to fully capture fine-scale spatial heterogeneities,
leading to artificially uniform concentrations within grid cells. This
oversimplification distorts the representation of non-linear aerosol
processes such as coagulation and gas-particle partitioning. Studies
have demonstrated that climate-relevant aerosol properties, including
aerosol optical properties \citep{gustafson_jr_downscaling_2011} and
CCN activity \citep{weigum_effect_2016}, are highly sensitive to model
resolution. Much of the sub-grid-scale variability arises from
spatially heterogeneous emissions \citep{qian_investigation_2010}, yet
many climate models fail to resolve this variability adequately.

Prior research on sub-grid variability of aerosol properties has
typically compared coarse-resolution global climate models (50–100 km)
against higher-resolution simulations (1–10 km)
\citep{qian_investigation_2010, gustafson_jr_downscaling_2011,
  weigum_effect_2016, crippa_impact_2017,
  lin_quantification_2017}. While increasing model resolution improves
the representation of emission heterogeneities, unresolved spatial
variability persists at sub-kilometer scales. Additionally,
climate models rely on Reynolds-averaged Navier-Stokes
parameterizations to represent boundary-layer turbulence, which fail
to capture the full complexity of turbulent transport and its
influence on aerosol processes.

Another approach to addressing sub-grid heterogeneity has been to use
LES, which explicitly captures fine-scale turbulent mixing and
chemical segregation.  \citet{brasseur_segregation_2023} review LES
applications investigating turbulence-chemistry interactions in
spatially heterogeneous environments.  Many LES studies have focused
on gas-phase chemistry, particularly the oxidation of reactive
volatile organic compounds (VOCs) such as isoprene, demonstrating that
spatially heterogeneous emissions contribute to chemical segregation
between reactive species in the boundary layer
\citep{ouwersloot_segregation_2011, kaser_chemistry-turbulence_2015}.
Given the coupling between gas and aerosol phases through gas-particle
partitioning, it is likely that chemical segregation due to spatially
heterogeneous emissions also influences aerosol properties. While some
LES-based studies have incorporated aerosols, they have largely relied
on simplified treatments. Recent efforts have coupled
turbulence-resolving frameworks with aerosol models, such as the
Sectional Aerosol Model for Large Scale Applications (SALSA)
\citep{kokkola_salsa_2008} with UCLALES
\citep{tonttila_uclalessalsa_2017}, the Parallelized Large-Eddy
Simulation Model (PALM) \citep{kurppa_implementation_2019}, and the
modal aerosol model M7 \citep{vignati_m7_2004} with the Dutch
Atmospheric Large-Eddy Simulation model (DALES)
\citep{de_bruine_explicit_2019}. However, despite their
high-resolution transport schemes, these models rely on relatively
coarse aerosol representations. For example, UCLALES-SALSA employs a
10-bin sectional scheme, while DALES modifies the seven-mode M7 model
to incorporate additional hydrometeor modes. To our knowledge, no
turbulence-resolving model has yet incorporated a particle-resolved
aerosol treatment, which would enable a fully detailed representation
of aerosol composition, properties, and aging alongside LES-resolved
turbulent transport.

This study aims to analyze the coupled effects of spatial
heterogeneity in surface emissions (including both gas-phase species
and primary aerosols), aerosol aging processes, and their impact on
CCN activity. To achieve this, we conduct the first particle-resolved
LES simulations, establishing a high-resolution aerosol-transport
modeling framework that explicitly represents both turbulent transport
and aerosol composition.

This paper is organized in the following manner. Section 2 presents
the modeling framework, WRF-PartMC-MOSAIC-LES, along with a
description of emissions scenarios with varying spatial
heterogeneity. Section 3 discusses the results of simulation runs,
including changes in aerosol size distribution, composition,
hygroscopicity, and CCN activity. We conclude with remarks on the
implications of this study, limitations stemming from its idealized
nature, and potential directions for future work.

\section{Methods}

\subsection{WRF-PartMC-MOSAIC-LES}

The aerosol-transport model WRF-PartMC-MOSAIC-LES integrates multiple
sub-models responsible for transport, aerosol representation, and
multiphase chemistry. It is extends the aerosol-transport column model
WRF-PartMC, originally developed by \citet{curtis_single-column_2017}
and later extended to include advection
\citep{gmd-17-8399-2024}. WRF-PartMC couples the Weather Research and
Forecasting model (WRF) \citep{skamarock_description_2008} with the
particle-resolved aerosol model PartMC
\citep{riemer_simulating_2009}. PartMC represents a population of
aerosol particles using an ensemble of computational particles, each
assigned a weight (i.e., particle multiplicity) to capture the
diversity of particle sizes and composition observed in real-world
aerosol populations. As particles age, their composition evolves
dynamically. Since PartMC operates as a box model it does not track
the spatial position of individual particles within a computational
grid cell. Instead, particle transport is handled by a stochastic
advection algorithm that interfaces WRF's dynamical core
\citep{gmd-17-8399-2024}. WRF-PartMC has been used in one-dimensional
simulations to resolve vertical gradients in aerosol composition and
mixing state \citep {curtis_single-column_2017} and, more recently. in
three-dimensional simulations to study aerosol transport within a
regional domain driven by by simulated meteorology
\citep{gmd-17-8399-2024}.

LES models explicitly resolve large-scale turbulent motion; however,
they must parameterize sub-grid eddies at scales below the grid
resolution (typically 10--100 m) as well as the down-gradient tendency
of turbulent kinetic energy (TKE) from large to small scales, where it
ultimately dissipates as heat. This requires the use of closure
schemes. In WRF-PartMC-MOSAIC-LES, turbulent mixing is parameterized
using Deardorff's TKE scheme for eddy diffusivity and eddy viscosity
\citep{deardorff_stratocumulus-capped_1980}.

Gas-phase and aerosol chemistry are represented using the Model for
Simulating Aerosol Interactions and Chemistry (MOSAIC)
\citep{zaveri_model_2008}. MOSAIC is comprised of multiple sub-models,
including the Carbon Bond Mechanism version Z (CBM-Z) which solves gas
phase chemistry \citep{zaveri_new_1999}. Phase-dependent partitioning
of aerosol species is handled by the Multicomponent Equilibrium Solver
for Aerosols (MESA) \citep{zaveri_computationally_2005}. Activity
coefficients of electrolytes are parameterized via the multicomponent
Taylor expansion method (MTEM) \citep{zaveri_new_2005}. To efficiently
solve the numerically stiff set of solid-liquid phase reactions,
MOSAIC employs the adaptive step time-split Euler method (ASTEM)
\citep{zaveri_model_2008}. MOSAIC models aerosol chemistry for both
inorganic and organic compounds such as nitrate, ammonium, sulfate,
black carbon (BC), and a limited set of secondary organic aerosol
(SOA) species.

\subsection{Computational domain setup}

The computational domain extends 10~km in both the $x$- and
$y$-directions, with a horizontal grid spacing of 100~m.  Vertically,
the domain reaches 2~km and is represented with 200~vertical
levels. WRF uses an $\eta$ vertical coordinate system, and for LES
runs, vertical levels are linearly spaced in pressure. This results in
an effective vertical resolution of approximately 10 m. Simulations
begin on the Vernal Equinox at 09:00 local time and conclude at 15:00
for a total duration of 6 hours to maintain balanced photolysis rates
throughout the simulation period. Each grid cell is initialized with
100 computational particles, resulting in 100 million total
particles. As processes such as emission, transport, and coagulation
modify the number concentration of particles within grid cells, the
total number of computational particles is dynamically
adjusted---doubling when it falls to half the initial value and
halving when it reaches twice the initial value---to maintain
computational efficiency.

The domain surface is flat and uniform without topographical features
or land-use variations. WRF-PartMC-MOSAIC-LES is not coupled to one of
WRF's radiation sub-models. Instead, MOSAIC employs idealized
parameterizations to determine photolysis rates based on the solar
zenith angle. Due to the lack of a radiation sub-model, surface
heating is imposed uniformly across the domain using a constant rate
of 0.24~$\rm K \, m^{-1} \, s^{-1}$.

Initial conditions and emissions for both aerosol and gas phase are
representative of an urban environment and are based on
\citet{riemer_simulating_2009}. The initial concentrations and
emission rates were adapted from the 1987 Southern California Air
Quality Study (SCAQS) during which measurements of gas phase species
and particulate matter mass concentrations were collected at multiple
sites across the Los Angeles basin \citep{zaveri_model_2008}. Table
\ref{table:gas_emiss_ics} provides initial concentrations and emission
rates for gas phase species, while Table \ref{table:aero_emiss_ics}
details aerosol initial conditions and emission rates categorized by
aerosol modes. Initially, the aerosol consists of an equal mixture of
ammonium sulfate and primary organic aerosol (POA). The three emission
modes representing cooking and vehicular combustion comprise varying
mixtures of POA and BC. To allow for simulation spin-up and the full
development of the convective boundary layer, emissions are set to
zero during the first hour of simulations. Thereafter, emitted
compounds are released at the surface at constant rates as specified
by Tables \ref{table:gas_emiss_ics} and \ref{table:aero_emiss_ics} for
the remainder of simulations.


\begin{table}[!t]
\centering
\caption{Gas phase emissions and initial conditions. Table adapted from \citet{riemer_simulating_2009} with permission.}
\begin{tabular*}{\linewidth}{@{\extracolsep{\fill}} lccr}
\\[-2ex]\hline 
     \hline \\[-2ex] Species & Symbol & Initial Mole Fraction (ppb) & Emissions (nmol m\textsuperscript{-2} s\textsuperscript{-1}) \\
\midrule
Nitric oxide & NO & 0.1 & 31.8 \\
Nitrogen dioxide & NO\textsubscript{2} & 1.0 & 1.67 \\
Nitric acid & HNO\textsubscript{3} & 1.0 & \\
Ozone & O\textsubscript{3} & 50.0 & \\
Hydrogen peroxide & H\textsubscript{2}O\textsubscript{2} & 1.1 & \\
Carbon monoxide & CO & 21 & 291.3 \\
Sulfur dioxide & SO\textsubscript{2} & 0.8 & 2.51 \\
Ammonia & NH\textsubscript{3} & 0.5 & 6.11 \\
Hydrogen chloride & HCl & 0.7 & \\
Methane & CH\textsubscript{4} & 2200 & \\
Ethane & C\textsubscript{2}H\textsubscript{6} & 1.0 & \\
Formaldehyde & HCHO & 1.2 & 1.68 \\
Methanol & CH\textsubscript{3}OH & 0.12 & 0.28 \\
Methyl hydrogen peroxide & CH\textsubscript{3}OOH & 0.5 & \\
Acetaldehyde & ALD2 & 1.0 & 0.68 \\
Paraffin carbon & PAR & 2.0 & 96 \\
Acetone & AONE & 1.0 & 1.23 \\
Ethene & ETH & 0.2 & 7.2 \\
Terminal olefin carbons & OLET & 2.3 \(\cdot 10^{-2}\) & 2.42 \\
Internal olefin carbons & OLEI & 3.1 \(\cdot 10^{-4}\) & 2.42 \\
Toluene & TOL & 0.1 & 4.04 \\
Xylene & XYL & 0.1 & 2.41 \\
Lumped organic nitrate & ONIT & 0.1 & \\
Peroxyacetyl nitrate & PAN & 0.8 & \\
Higher organic acid & RCOOH & 0.2 & \\
Higher organic peroxide & ROOH & 2.5 \(\cdot 10^{-2}\) & \\
Isoprene & ISOP & 0.5 & 0.23 \\
Alcohols & ANOL & & 3.45 \\
\\[-2ex]\hline 
     \hline \\[-2ex]
\end{tabular*}
\label{table:gas_emiss_ics}
\end{table}


\begin{table}[!t]
\centering
\caption{Aerosol emissions and initial conditions. Table adapted from \citet{riemer_simulating_2009} with permission.}
\begin{tabular*}{\linewidth}{@{\extracolsep{\fill}} cccccc}
\\[-2ex]\hline 
     \hline \\[-2ex] Initial/Background  & $N$ (m$^{-3}$) & $D_{\text{gn}}$ ($\mu$m) & $\sigma_g$ & Composition by Mass\\
 \midrule
Aitken Mode & $3.2 \cdot 10^9$ & 0.02 & 1.45 & 50\% (NH$_4$)$_2$SO$_4$, 50\% POA\\
Accumulation Mode & $2.9 \cdot 10^9$ & 0.116 & 1.65 & 50\% (NH$_4$)$_2$SO$_4$, 50\% POA\\
\midrule
Emissions & $E$ (m$^{-2}$ s$^{-1}$) & $D_{\text{gn}}$ ($\mu$m) & $\sigma_g$ & Composition by Mass\\
\midrule
Meat cooking & $9 \cdot 10^6$ & 0.086 & 1.9 & 100\% POA\\
Diesel vehicles & $1.6 \cdot 10^8$ & 0.05 & 1.7 & 30\% POA, 70\% BC \\
Gasoline vehicles & $5 \cdot 10^7$ & 0.05 & 1.7 & 80\% POA, 20\% BC \\
\\[-2ex]\hline 
     \hline \\[-2ex]
\end{tabular*}
\label{table:aero_emiss_ics}
\end{table}

\textcolor{red}{Insert tilde between number and unit everywhere to protect space.}

Meteorological initial conditions are specified using an idealized
convective boundary layer sounding, where the surface is 5~K warmer
than the mixing layer, and an inversion of 8 K caps the layer at 1
km. %\hl{Maybe include a figure here?}  The wind profile is set to
zero throughout the domain.

%\hl{putting some namelist parameters here for time being for convenience}
%\begin{lstlisting}
% &physics
% mp_physics                          = 0,     0,     0, ! microphysics scheme
% ra_lw_physics                       = 0,     0,     0, ! long wave radiation scheme
% ra_sw_physics                       = 0,     0,     0, ! short wave radiation scheme
% radt                                = 0,     0,     0, ! radiation time step
% sf_sfclay_physics                   = 0,     1,     1, ! surface layer parameterization
% sf_surface_physics                  = 0,     0,     0, ! surface sub model
% bl_pbl_physics                      = 0,     0,     0, ! boundary layer parameterizations
% bldt                                = 0,     0,     0, !  boundary layer scheme time delta
% cu_physics                          = 0,     0,     0, ! cumulus physics model
% cudt                                = 0,     0,     0, ! cumulus physics scheme time delta
% isfflx                              = 2, ! surface heat and moisture flux
% num_land_cat = 24,
% num_soil_layers                     = 5, ! does this matter since I don't use a surface model?
%\end{lstlisting}


\subsection{Emissions Scenarios}

\begin{figure}[!t]
	\centering
	\includegraphics[]{figures/SH-scenarios.pdf}
	\caption{Emissions spatial heterogeneity scenarios. The pattern of emissions is shown as a cross section of the x-y plane at ground level. Shaded areas correspond to regions of emissions. The hue of shading indicates the intensity of emissions scaling ranging from light blue (low emissions scaling) to dark blue (high emissions scaling). Both the spatial heterogeneity metric $\eta$ and the fraction of area covered by emissions are displayed in the bottom of each scenario.}
	\label{fig:sh-scenarios}
\end{figure} 

To assess the impact of emission spatial heterogeneity on aerosol
properties, we examine multiple emission scenarios, shown in Figure
\ref{fig:sh-scenarios}. The first scenario, referred to as the uniform
base case, distributed emissions evenly across the entire domain. This
serves as a proxy for coarser-resolution models that do not resolve
spatial heterogeneity of emissions and instead assume uniform
emissions of gases and primary aerosols across grid cells. The
remaining scenarios introduce increasesing levels of spatial
heterogeneity, allowing for direct comparison against the uniform base
case to evaluate its influece to aerosol properties. Scenario 1
represents an idealized urban-rural interface, where emissions are
released in half the domain with no emissions occurring in the other
half. Scenario 2 contains a narrow strip of emissions running through
the center of the domain, corresponding to an emission pattern typical
of a major roadway. Lastly, scenario 3 places all emissions in a
single grid cell in the domain center representing a point source such
as an industrial plume.

These scenarios span a range of spatial heterogeneity, which is
quantified using the metric $\eta$ developed by
\citep{mohebalhojeh_2024} \hl{make sure to update this citation when
  Matin's paper is published}. The metric $\eta$ is a normalized
measure of spatial heterogeneity, ranging from 0 (completely
homogeneous) to 1 (maximally heterogeneous). For a discrete
2-dimensional scalar field $f$ over a domain $S$ with
dimensions $N$ by $M$, $\eta$ is defined as

\begin{equation}
\eta(f, S) = \frac{\sum_{\tilde{S}\in \mathbb{R}}|\overline{f}(S) - \overline{f}(\tilde{S})|}{\overline{f}(S)\left[\frac{3}{2}(N\times M)(N-1)(M-1) + N(N-1) + M(M-1)\right]}, 
\end{equation}
where $\overline{f}(S)$ is the domain mean and
$\overline{f}(\tilde{S})$ is the mean over a subset of the domain
$\tilde{S}$. Thus, the metric is computed by summing the absolute
value of the difference between the domain mean $\overline{f}(S)$ and
all domain subset means $\overline{f}(\tilde{S})$, normalized by the
product of the domain mean and the number of possible
subsets. \citet{mohebalhojeh_2024} show that the metric is
translationally invariant when the scalar field $f$ is shifted within
$S$. Furthermore, they prove that the
maximum spatial heterogeneity occurs when the
scalar field is zero everywhere except at a single point
where it takes the value $MN\times\overline{f}(S)$.

Thus, the uniform base case corresponds to the homogeneous condition
($\eta = 0$) while scenario 3---the point-source
emissionn---represents the maximally heterogeneous case ($\eta =
1$). To ensure consistent mass emissions across scenarios, emission
rates are scaled by the fraction of area covered by emissions. For
instance, in scenario 3, this results in a scaling of $10,000$
($M=N=100$) for the point-source emission. The fraction of area
covered by emissions is displayed for each scenario alongside the
spatial heterogeneity values in Figure \ref{fig:sh-scenarios}.

\section{Results}

\textcolor{red}{Is it emissions spatial heterogeneity or emission ...?}

Key findings of this paper are structured as follows. Section
\ref{sec:gas-impacts} investigates the relationship between emission
spatial heterogeneity and the gas phase. In Section
\ref{sec:size-dist}, impacts of emissions spatial heterogeneity on
bulk aerosol properties (number and mass concentrations) are
presented. Impacts on aerosol composition are then presented in
Section \ref{sec:aero-comp} with particular focus on sulfate, ammonium,
and nitrate, as these compounds play a key role in particle
hygroscopicity. Leveraging the particle-resolved framework, we
further evaluate how particle hygroscopicity responds to emissions spatial
heterogeneity.

Building on these results, Section \ref{sec:ccn-activ} investigates
how emissions spatial heterogeneity impacts CCN activity across a
range of supersaturation levels ranging ($0.1\%$ to $1.0\%$). Lastly,
we explore the impact of two key factors that govern the impact of
emissions spatial heterogeneity on CCN activity. Namely, we
investigate impacts of aerosol composition---in particular, the
presence of ammonia---and mean wind in the planetary boundary layer in
Sections \ref{sec:influence-ammonia} and \ref{sec:influence-wind},
respectively.

\textcolor{red}{I think we said that we would not do the wind part.}

\subsection{Impacts of emissions spatial heterogeneity on gas phase species}\label{sec:gas-impacts}

\begin{figure}[!h]
	\centering
	\includegraphics[]{figures/gas-spatial-heterogeneity-time36-z45.pdf}
	\caption{Cross sections in the x-y plane of gas phase species NH$_3$ (left column), HNO$_3$ (center column), and OH (right column). Cross sections are shown at a height of approximately $z\approx900$ m and at $t=6$ h. Rows are organized by emissions scenario ranging from the uniform base case to scenario 3. Color shading indicates the intensity of concentrations for each species and is normalized by the maximum and minimum concentration observed in each cross section. The value of the spatial heterogeneity metric $\eta$ is displayed alongside each cross section plot.}
	\label{fig:gas-cross-sec}
\end{figure} 

Figure \ref{fig:gas-cross-sec} displays x-y cross sections of ammonia,
nitric acid, and the hydroxyl radical (OH) at $t=6$ h taken at $z\approx900$~m in the upper boundary layer.

\textcolor{red}{{\it All this about how the figure is organized should be deleted. The reader can figure this out by looking at the figure/caption. What should go here is what it means.}
Cross sections are organized
by species along columns and by emissions scenario along rows. The
spatial heterogeneity $\eta$ of each cross section is shown at the top
of each subplot. The concentration of compounds in each cross section
is indicated by colored shading, ranging from dark blue (low
concentrations) to yellow (high concentrations). Note that in order to
illustrate the structure of each cross section, the coloring does not
indicate the absolute magnitude of a given species concentration as
the absolute difference between, for instance, ammonia concentrations
near the emission plume and OH concentrations span orders of
magnitude. Instead, the colored shading of each cross section is
normalized relative to the range of concentrations observed.}

Among the three species, ammonia exhibits the highest spatial
heterogeneity with $\eta$ values 3--4 times higher than those of
nitric acid and OH. This is primarily because ammonia is emitted while
nitric acid and OH are formed due to chemical
reactions. \textcolor{red}{Can we explain a little more why this would
  be the case?} The similar structure between ammonia and nitric acid
indicates that nitric acid formation is most effective over the
emissions plume due to the emission of NO and NO$_2$. The spatial
pattern of OH follows that of ammonia and nitric acid but with an
inverse relationship: OH concentrations are lower near the center of
the emissions plume and higher concentrations farther away. This
indicates rapid oxidation of many reactive compounds occurring in the
emissions plume. Consequently, OH becomes spatially segregated from
its reactants in the emissions plume as OH in the immediate vicinity
of the plume is consumed.

\begin{figure}[!h]
	\centering
	\includegraphics[]{figures/aerosol-gas-vertical-profiles-time36.pdf}
	\caption{Vertical profiles of gas phase species NH$_3$ (left), HNO$_3$ (center), and OH (right) at $t=6$ h. For NH$_3$ (left) and HNO$_3$, the mean value is displayed at each vertical level. For OH, the 5th percentile at each vertical level is shown to indicate the local changes to the OH concentration near the emissions plume. Values for the uniform base case are shown as a solid black line while emissions scenarios 1--3 are shown as colored solid lines.}
	\label{fig:gas-profiles}
\end{figure} 

Figure \ref{fig:gas-profiles} presents vertical profiles of ammonia,
nitric acid, and OH. For ammonia and nitric acid, the profiles show
horizontal averages at each vertical level across the entire domain.
As emissions spatial heterogeneity increases, both ammonia and nitric
acid concentrations decrease on average. As will demonstrated in
Section \ref{sec:aero-comp}, this is due to increased partitioning of
these species into the aerosol phase to form ammonium nitrate.

For OH, the vertical profiles show the 5th percentile of
concentrations within each level, rather than the domain-wise mean, to
better capture localized depletion nea the emissions plume. Scenarios
with high emissions spatial heterogeneity result a pronounced
reduction in OH near the emissions plume. Whereas in the uniform base
case, the level-mean concentration of OH is 0.61~pptv at $z=1$~km,
whereas in scenario~3 (the most heterogeneous case) they drop by 54\%
to 0.28~pptv.

\subsection{Aerosol size distributions}\label{sec:size-dist}

\begin{figure}[!h]
	\centering
	\includegraphics[]{figures/combined_num_mass_conc_i50_j50_k60.pdf}
	\caption{Number (left) and mass (right) distributions for each emissions scenario in the upper boundary layer ($z=800$) m and $t=6$ h. The initial condition is shown as the dashed black line. Values for the uniform base case are shown as a solid black line while emissions scenarios 1--3 are shown as colored solid lines.}
	\label{fig:size-dists}
\end{figure} 

Number and mass distributions for each emissions scenario are shown in
Figure \ref{fig:size-dists}. Each size distribution is taken from a
vertical level in the upper boundary layer at $z\approx800$
m. \textcolor{red}{This needs to be expressed a bit differently. You
  made this postprocessing choice to reduce noise.} Due to the
stochastic treatment of aerosol particles in WRF-PartMC and the
selected number of computational particles per grid cell ($N = 100$),
both number and mass distributions represent the average distribution
in a 1 km$^2$ region centered over the emissions plume (i.e., size
distributions are averaged over a $10\times10$ grid cell region). For
the uniform base case, scenario 2 and scenario 3, this region is
directly over the center of the domain. For scenario 1, emissions are
released in one half of the domain that is offset from the center, and
thus the averaging region is located in the center of the emissions
patch. For each size distribution, data have been binned into 100
bins, ranging in size from $10^{-9}$ to $10^{-3}$ m.

To quantify changes in particle populations, the nimber and mass
concentration of Aitken ($D_p < 50$~nm) and accumulation ($D_p <
50$~nm) mode particles were calculated. As the spatial heterogeneity
of emissions increases, \textcolor{red}{{\it eliminate the ``we find
    that'' phrase.} we find that} the number of Aitken mode particles
decreases by up to 81\% while the number of accumulation mode
particles increases by up to 246\%. This points to enhanced Brownian
coagulation among ultra-fine particles as a result of higher local
concentrations near the emissions plume core. \textcolor{red}{
  Brownian motion is responsible for the decrease, but not for the
  increase? I think the increase is because of the growth due to
  nitrate formation}

Similarly, the mass distribution of Aitken mode particles decreases by
up to 74\% for high emissions spatial heterogeneity scenarios, while
the accumulation mode mass fraction increases by up to
309\%. Coagulation of smaller Aitken mode particles with accumulation
mode particles contributes little change in the mass distribution as
indicated by a slight reduction in the Aitken mode mass
concentration. Rather, the increase in mass concentration in the
accumulation mode is largely due to secondary aerosol formation via
gas-particle partitioning. This process is explored in further detail
in the next section.


\subsection{Aerosol composition}\label{sec:aero-comp}

\begin{figure}[!h]
	\centering
	\includegraphics[]{figures/aerosol-SNA-spatial-heterogeneity-time36-z45.pdf}
	\caption{Cross sections in the x-y plane of aerosol species NH$_4^+$ (left column), NO$_3^-$ (center column), and SO$_4^{2-}$ (right column). Cross sections are shown at a height of approximately $z\approx900$ m and at $t=6$ h. Rows are organized by emissions scenario ranging from the uniform base case to scenario 3. Color shading indicates the intensity of concentrations for each species and is normalized by the maximum and minimum concentration observed in each cross section. The value of the spatial heterogeneity metric $\eta$ is displayed alongside each cross section plot.}
	\label{fig:aero-cross-sec}
\end{figure} 

Figure \ref{fig:aero-cross-sec} shows cross sections of aerosol
ammonium, nitrate, and sulfate in the upper boundary layer
($z\approx900$~m) at $t=6$~h. The spatial distribution of both
ammonium and nitrate closely follows emission plume, indicating that
ammonium nitrate formation occurs where ammonia and nitric acid are
abundant. In these regions excess gas phase concentrations drive
partitioning into the aerosol phase. Away from the emissions plume,
the ammonium and nitrate concentrations decrease as the equilibrium
condition shifts to the gas phase.

In contrast, sulfate is more uniformly distributed across all
scenario, resulting in lower spatial heterogeneity ($\eta=0.021$ for
scenario 3) compared to ammonium ($\eta = 0.142$) and nitrate
($\eta=0.739$).  This is due to the extremely low volatility of
sulfuric acid which remains almost entirely in the aerosol phase as
sulfate. Unlike ammonia and nitric acid, which partition dynamically
between gas and aerosol phases, sulfate does not re-enter the gas
phase. Consequently, the spatial distribution of aerosol species is
determined by their volatility, with less volatile species exhibiting
more uniform distributions, while higher volatility species remain
concentrated near plumes with high concentrations of corresponding gas
phase precursors.


\begin{figure}[!h]
	\centering
	\includegraphics[]{figures/aerosol-SNA-vertical-profiles-time36.pdf}
	\caption{Vertical profiles of aerosol species NH$_4^+$ (left), NO$_3^-$ (center), and SO$_4^{2-}$ (right) at $t=6$ h and in parts per billion by volume (ppbv). For each compound, the mean value is displayed at each vertical level. Values for the uniform base case are shown as a solid black line while emissions scenarios 1--3 are shown as colored solid lines.}
	\label{fig:vertical-profile-SNA}
\end{figure} 

Figure \ref{fig:vertical-profile-SNA} shows vertical profiles of
aerosol ammonium (NH$_4^+$), nitrate (NO$_3^-$), and sulfate
(SO$_4^{2-}$) for each emissions scenario. These profiles represent
the average concentrations within each vertical level at the end of
each simulation ($t=6$ h).

Sulfate concentrations are nearly uniform within the boundary layer
and rapidly decrease above the entrainment zone due to limited mixing
with the free troposphere and the boundary layer. Sulfate
concentrations decrease as the emissions spatial heterogeneity
increases. \textcolor{red}{{\it avoid the ``note that'' phrase. The
    next sentence is very long and not easy to understand. } Note that
  for scenarios with high emissions spatial heterogeneity, the
  elevated concentration of reactive gas phase compounds in the
  emissions plume such as volatile organic compounds (VOCs) alongside
  sulfate's gas phase precursor SO$_2$ results in greater effective
  competition for oxidation with OH.} This is linked to the depletion
of OH within the emissions plume, where volatile organic compounds
(VOCs) and SO$_2$ compete for oxidation.  With OH rapidly depleted
near the emissions plume, less is available \textcolor{red}{{\it use
    verbs instead noun constructs} for oxidation} to oxidize of SO$_2$
into H$_2$SO$_4$, thereby reducing sulfate formation. OH from outside
the emissions plume is not mixed fast enough into the plume to restore
its concentration (see Figure \ref{fig:gas-cross-sec}). The
segregation of OH and SO$_2$ due to emissions spatial heterogeneity
thus alters sulfate production and its gas-particle partitioning.

Both ammonium and nitrate concentrations increase with height in the
boundary layer due to the strong temperature dependence of ammonium
nitrate formation. Nitrate availability depends on the presence of
free ammonia, i.e., ammonia \textcolor{red}{in excess {\it avoid the
    construct ``of what is'' required}} not already neutralizing
sulfate as ammonium sulfate. In the lowest 500~m of the boundary
layer, the concentration of NH$_4^+$ decreases under high emissions
spatial heterogeneity due to lower
sulfate concentrations at higher emissions spatial
heterogeneity.

At higher altitudes ($z\sim1.2$ km), ammonium nitrate formation is
enhanced under high emissions spatial heterogeneity.
\textcolor{red}{{\it Don't put so many words before the verb. That's what Germans do, and english speakers don't like it,
    so I have been told many times.} The decrease in sulfate
  concentrations for scenarios with high emissions spatial
  heterogeneity results in higher concentrations of free ammonia, thus
  allowing the formation of more ammonium nitrate.}  The reduction in
sulfate increases free ammonia, allowing more nitrate to
form. \textcolor{red}{{\it In the next sentence, the fact that there
    is almost no nitrate formed is I believe more because of the
    overall lower concentrations of nitric acid and ammonia, not
    because the ammonia was taken by the sulfate}} In the uniform base
case, almost no nitrate is formed due to the lack of free ammonium as
nearly all NH$_4^+$ is bound to sulfate as ammonium sulfate. This
indicates the strong dependence of nitrate concentrations on the
composition of the aerosol and the level of emissions spatial
heterogeneity.


\begin{figure}[!h]
	\centering
	\includegraphics[]{figures/speciated-mass-frac-three-panel-z40.pdf}
	\caption{Speciated mass fraction as a function of particle diameter for emission scenario extremes. The initial condition is shown on the left, indicating that aerosol begin as an equal mixture of OC and ammonium sulfate. On the right, emissions scenarios with minimum spatial heterogeneity (top, uniform base case) and maximum spatial heterogeneity (bottom, scenario 3) are shown after 6 hours.}
	\label{fig:speciated-mass-frac}
\end{figure} 

Figure \ref{fig:speciated-mass-frac} shows the size-resolved mass
fraction of aerosols in the upper boundary layer ($z\approx 800$ m)
for the initial condition and at the end of simulations ($t=6$ h) for
both the uniform base case and scenario 3 (highest emissions spatial
heterogeneity). After 6 hours, significan differences in composition
emerge. Under uniform emissions, particles mainly consist of BC and
organic carbon (OC) along with some sulfate. In contrast, under
scenario 3, particles are dominated by nitrate, ammonium, and sulfate,
which together comprise 50--80\% of aerosol mass.

The CCN activity of particles in the size range of 50--100 nm is
largely governed by their composition. \textcolor{red}{{\it I'm
    thinking the next couple of sentences are a bit too basic to
    mention explicitly} Hygroscopic species enhance activation by
  lowering the critical supersaturation via the solute effect. Without
  the aid of hygroscopic material, small aerosol particles possess
  high critical supersaturations due to the Kelvin effect which
  increases vapor pressure over the particle surface due to its
  curvature.} Figure \ref{fig:speciated-mass-frac} indicates that, for
the uniform base case, particles in this size range are primarily
composed of BC and \textcolor{red}{{\it we should call OC POA}} OC,
which have low hygroscopicity. Conversely, in scenario 3, sulfate,
nitrate, and ammonium dominate making particles more hygroscopic. This
suggests that particles in the size range, whose CCN activity depends
on aerosol composition, exhibit greater hygroscopic under high
emissions spatial heterogeneity scenarios, allowing them to activate
at lower supersaturations.

\begin{figure}[!t]
	\centering
	\includegraphics[]{figures/2d-kappa-dist-three-panel-z40.pdf}
	\caption{2-dimensional number distributions $n(D_p, \kappa)$ for emission scenario extremes. The initial condition is shown on the left, indicating that all aerosol begin as internally mixed particles with uniform $\kappa$. On the right, emissions scenarios with minimum spatial heterogeneity (top, uniform base case) and maximum spatial heterogeneity (bottom, scenario 3) are shown after 6 hours. Cell coloring indicates particle number concentration. Black solid contours indicate supersaturation in $\%$. Particles to the right of a contour line activate at the indicated supersaturation.}
	\label{fig:kappa-dist}
\end{figure} 

Figure \ref{fig:kappa-dist} shows 2-dimensional number distributions
$n(D_p, \kappa)$ as a function of particle diameter $D_p$ and particle
hygroscopicity parameter $\kappa$ in the upper boundary layer
($z\approx 800$~m). For the initial condition all particles possess
the same composition and thus the same hygroscopicity.  The right
panel compares distributions at $t=6$~h for the uniform base case and
scenario 3.

In the uniform base case, two distinct particle groups emerge: one
with low $\kappa$ values (0--0.3), corresponding to primary
carbonaceous aerosols that have not undergone significant aging, and
another with higher $\kappa$ (0.3-0.6), representing particles that
have undergone coagulation and gas-particle partitioning. The latter
group is enriched in sulfate as seen in Figure
\ref{fig:speciated-mass-frac}.

Under scenario 3, particles exhibit significantly higher
hygroscopicities. For instance, the hygroscopicity of particles with
diameter of 100 nm exceeds $\kappa>0.6$ (indicating highly hygroscopic
particles), whereas $\kappa$ only reaches up to 0.4 in the uniform
base case. These resultss show that spatially heterogeneous emissions
can elevate the hygroscopicity of particles whose CCN activity is
dependent on particle composition ($D_p\sim50\text{--}100$ nm).

Differences in the $\kappa$ distributions between the uniform base
case and scenario 3 stem from the interaction between emissions
spatial heterogeneity and sub-grid scale aerosol processes. Enhanced
coagulation in emissions plumes reduces Aitken mode particle
concentrations, explaining the absence of a low-$\kappa$ carbonaceous
aerosol group in scenario 3. Furthermore, spatially heterogeneous
emissions promote gas-particle partitioning increasing the
hygroscopicity. In particular, the formation of ammonium nitrate in
scenario 3 due to reduced levels of sulfate and a corresponding
increase in free ammonia elevate particle
hygroscopicity. \textcolor{red}{{\it again, in the last sentence, I
    agree that ammonium nitrate makes particles more hygroscopic, but
    I think the more nitrate comes from the higher precursor
    concentrations.}


\subsection{CCN activity}\label{sec:ccn-activ}

\begin{figure}[!h]
	\centering
	\includegraphics[]{figures/aerosol-ccn-vertical-profiles-time36.pdf}
	\caption{Vertical profiles for CCN concentrations activating at supersaturations $S=0.1, 0.3, 0.6, 1.0\%$ and at $t=6$ h. Concentrations are displayed in number of CCN per kilogram of dry air and are scaled by a factor of $1\times10^{-9}$. Values for the uniform base case are shown as a solid black line while emissions scenarios 1--3 are shown as colored solid lines.}
	\label{fig:ccn-vertical-prof}
\end{figure} 

Figure \ref{fig:ccn-vertical-prof} shows vertical profiles of the CCN
number \textcolor{red}{{\it should this be CCN number per kilogram of
    dry air?} concentration per kilogram of dry air} for
supersaturations $S$ ranging from $S=0.1\%$ to $S=1.0\%$ across
different emissions scenario. Since the ambient relative humidity (RH)
never exceeds 100\% in these simulations, the reported CCN
concentrations represent the number of particles that would activate
if RH were raised to the specified supersaturation.

As noted earlier, emissions spatial heterogeneity influences aerosol
processes such as coagulation and gas-particle partitioning, altering
the particle number, size, composition, and hygroscopicity. These
modifications, in turn, impact CCN activity, though the dominant
processes and their effects vary with supersaturation.

At lower supersaturations ($S=0.1\text{--}0.3\%$), CCN
\textcolor{red}{{\it I think this should be concentration rather than
    activity} activity} increases with emissions spatial heterogeneity
in the upper boundary layer. This is due to enhanced formation of
ammonium nitrate in the cooler, sulfate-poor environment, which
increases activation of ultrafine particles in the range of 50--100 nm
due to the high hygroscopicity of ammonium nitrate.

At higher supersaturations ($S=0.6\text{--}1.0\%$), CCN concentrations
still increase in the upper boundary layer for scenarios with lower
spatial heterogeneity, however, in scenario 3, the CCN concentrations
decrease, particularly at $S=1.0\%$, where CCN concentrations drop
below all other scenarios, including the uniform base case.

This occurs because enhanced coagulation in highly heterogeneous
emissions scenarios reduces the number of smaller particles that would
otherwise activate at high supersaturations.  Therefore, at
sufficiently high supersaturation and emissions spatial heterogeneity,
the negative effect on CCN concentration due to coagulation offsets
the positive effect of gas-particle partitioning of hygroscopic
material.

\begin{figure}[!h]
	\centering
	\includegraphics[]{figures/height-time-ccn-pdiff-multi-scenario.pdf}
	\caption{Time-height plots for the percent difference between CCN concentrations in the uniform base case and each emissions scenario and supersaturation level. Scenarios are organized by column. The supersaturation of CCN activation is organized by row. Red indicates an increase in CCN relative to the base case while blue indicates a reduction in CCN concentrations. Contour lines indicating regions of constant percent difference are drawn on each panel in increments of 5\%.}
	\label{fig:time-height-ccn-pdiff}
\end{figure} 

Figure \ref{fig:time-height-ccn-pdiff} illustrates the temporal and
vertical evolution of the percent difference between the CCN
concentrations relative to the uniform base case for each scenario and
supersaturation level. The percent difference is calculated as
\begin{equation}
    \% \text{ difference} = 100\times\left(\frac{\overline{[\text{CCN}]}(t, z, S)_{\text{Scenario}} - \overline{[\text{CCN}]}(t, z, S)_{\text{Base case}}}{\overline{[\text{CCN}]}(t, z, S)_{\text{Base case}}}\right),
\end{equation}
where $\overline{[\text{CCN}]}(t, z,S)$ is the horizontally averaged
concentration of CCN at time $t$ and vertical level $z$ that activate
at supersaturation $S$.

The greatest increase in CCN concentration occurs for scenario 3,
\textcolor{red}{{\it no need to repeat that scenario 3 is the highest
    emission scenario after maybe the first time it appears in the
    paper. When we introduce the scenario nomenclature, we commit to
    using these names throughout the paper consistently.}}the highest
spatial heterogeneity emissions scenario, at $S=0.3\%$
\textcolor{red}{{\it punctuation -- there should be a comma here}},
where CCN concentrations increase by more than 25\% over $t=6$
h. Across all scenario, the location of greatest increase in CCN
\textcolor{red}{{\it rather than activity, we should simply say
    concentration} activity} grows with time following boundary layer
development.  Moving toward higher supersaturations and greater
emissions spatial heterogeneity in Figure
\ref{fig:time-height-ccn-pdiff}, the reduction in CCN activity due to
enhanced coagulation becomes evident after approximately 5 hours of
simulation. Notably, the regions of increased CCN concentrations
coincide with altitudes where shallow cumulus and stratiform clouds
tend to form. This suggests that emissions spatial heterogeneity could
enhance clould albedo through the first indirect effect.

%\begin{figure}[!t]
%	\centering
%	\includegraphics[]{figures/4panel-ccn-rel-err-vs-sh.pdf}
%	\caption{Make absolute value plots and remove coloring by height20}
%	%\label{fig:transport-vs-aerosol-model}
%\end{figure} 


\subsection{Influence of ammonia on aerosol composition and CCN activity}\label{sec:influence-ammonia}

To further explore the role of ammonia in CCN activity under spatially
heterogeneous emissions, we conducted additional simulations for the
uniform base case and scenario 3, setting total ammonium
(NH$_{3\text{, gas}} + $NH$_{4\text{, aerosol}}$) to zero. Emissions
of NH$_3$ are also set to zero to ensure that total ammonium remains
zero throughout each simulation.

\begin{figure}[!h]
	\centering
	\includegraphics[]{figures/aerosol-ccn-vertical-profiles-no-nh4-cases-time36.pdf}
	\caption{Vertical profiles for CCN concentrations in ammonia-free simulations that activate at supersaturations $S=0.1, 0.3, 0.6, 1.0\%$ and at $t=6$ h. Concentrations are displayed in number of CCN per kilogram of dry air and are scaled by a factor of $1\times10^{-9}$. Profiles for scenarios with ammonia are shown as solid lines while scenarios without ammonia are displayed as dashed lines for the uniform base case (black) and scenario 3 (chartreuse).}
	\label{fig:ccn-vertical-profile-no-ammonia}
\end{figure}

Figure \ref{fig:ccn-vertical-profile-no-ammonia} shows vertical
profiles of CCN mixing ratios at $t=6$ h for supersaturations ranging
from $S=0.1\%$ to $S=1.0\%$. Without ammonia, CCN concentrations at
each supersaturation level \textcolor{red}{{\it much closer with
    what?} agree much closer.} The peak of CCN concentrations in the
upper boundary layer and at lower supersaturations, previously
observed in scenario 3, disappears entirely. This underscores the
crucial role of ammonium nitrate formation in modulating CCN
concentrations under spatially heterogeneous emissions, especially at
lower supersaturations.

At higher supersaturations ($S=1.0\%$), scenarion 3 exhibits lower CCN
concentrations than its ammonia-free uniform base case
counterpart. This further confirms that, in the absence of
ammonia-driven gas-particle partitioning, coagulation-induced particle
loss dominates, leading to an overall reduction in CCN
concentration. The similarity in CCN profiles between the ammonia-free
cases of Scenario 3 and the uniform base case, apart from a slight
(~5\%) downward shift in Scenario 3 due to enhanced coagulation,
underscores the competing influences of emissions spatial
heterogeneity on aerosol-cloud interactions.

\subsection{Influence of PBL wind on CCN activity}\label{sec:influence-wind}

\conclusions  %% \conclusions[modified heading if necessary]

% work this text in somewhere in the conclusion
%Past research has established a clear link between the spatial heterogeneity of both gas phase and aerosol emissions, the processes by which the aerosol age, and their resulting climate-relevant properties including CCN activity which contribute to indirect radiative forcing. Simultaneously, there exists large uncertainty in the radiative forcing due to aerosols which results in part from the coarse representation of both aerosols and their transport in global scale models and their coupling with non-linear processes that occur at the sub-grid scale such as coagulation and turbulence-chemistry interactions. Past research has led to the development of detailed aerosol model treatments and transport representations, however there has yet to be a direct coupling between particle-resolved aerosol models and turbulence-resolving transport models for use in quantifying the effects of emissions spatial heterogeneity on the aerosol state and CCN activity.

This study investigates the impact of spatially heterogeneous
emissions on aerosol properties, including CCN activity, in a
convective boundary layer using the particle-resolved large-eddy
simulation modeling framework, WRF-PartMC-MOSAIC-LES. This
first-of-its-kind modeling platform enables a detailed process-level
analysis of the coupling between emissions spatial heterogeneity and
concentration-dependent aerosol processes such as coagulation and
gas-particle partitioning. To assess these interactions, we compare
multiple idealized emission scenarios against a base case of uniform
emissions, which serves as a proxy for coarser-resolved models that
lack the ability to resolve heterogeneity of emissions.

Our results demonstrate that emissions spatial heterogeneity
significantly alters key aerosol processes. In particular, nitrate
formation increases substantially in regions of high emissions
heterogeneity due to localized enhancements in nitric acid and ammonia
concentrations near the emissions plume core. This shifts the
equilibrium favoring ammonium nitrate formation in the aerosol
phase. Additionally, higher emissions heterogeneity intensifies
coagulation, accelerating particle growth and modifying the size
distribution of aerosols.

These aerosol process changes have downstream effects on CCN
activity. Notably, the influence of emissions spatial heterogeneity on
CCN concentrations is goverend by competing effects of coagulation and
gas-particle partitioning. Coagulation removes smaller particles that
would otherwise activate at high supersaturations, resulting in a
decrease in CCN activity at high supersaturations for scenarios with
high emissions spatial heterogeneity. Conversely, coagulation is not
as efficient at removing larger particles that activate at lower
supersaturations. In contrast, gas-particle partitioning results in an
increase of highly hygroscopic compounds such as ammonium nitrate
under high emissions spatial heterogeneity. As a result, CCN activity
at lower supersaturations ($S = 0.3\mbox{–-}0.6\%$) increases by up to
25\% in the upper boundary layer for emissions scenarios with high
spatial heterogeneity.

The of CCN activity to emissions spatial heterogeneity is highly
sensitive aerosol and gas phase composition. Given the key
contribution of ammonium nitrate formation in elevating CCN activity
under highly spatially heterogeneous scenarios, removing ammonia
weakens---or in some cases reverses---the trend between emissions
spatial heterogeneity and CCN concentrations.

This has important implcations for global climate models (GCMs), where
nitrate formation is often simplified by assuming equilibrium
partitioning or omitted altogether. Our findings highlight the
necessity of accurately representing nitrate in GCMs due to the strong
coupling between emissions spatial heterogeneity, aerosol composition,
and CCN activity. As model resolution continues to improve, advancing
the representation of aerosol chemistry will be critical to capturing
the full impact of emissions spatial heterogeneity on cloud
microphysics and climate.

%\begin{itemize}
%\item Limitations and future work
%\end{itemize}

%% The following commands are for the statements about the availability of data sets and/or software code corresponding to the manuscript.
%% It is strongly recommended to make use of these sections in case data sets and/or software code have been part of your research the article is based on.

\codeavailability{TEXT} %% use this section when having only software code available


\dataavailability{TEXT} %% use this section when having only data sets available


\codedataavailability{TEXT} %% use this section when having data sets and software code available


\sampleavailability{TEXT} %% use this section when having geoscientific samples available


\videosupplement{TEXT} %% use this section when having video supplements available


\appendix
\section{}    %% Appendix A

\subsection{}     %% Appendix A1, A2, etc.


\noappendix       %% use this to mark the end of the appendix section. Otherwise the figures might be numbered incorrectly (e.g. 10 instead of 1).

%% Regarding figures and tables in appendices, the following two options are possible depending on your general handling of figures and tables in the manuscript environment:

%% Option 1: If you sorted all figures and tables into the sections of the text, please also sort the appendix figures and appendix tables into the respective appendix sections.
%% They will be correctly named automatically.

%% Option 2: If you put all figures after the reference list, please insert appendix tables and figures after the normal tables and figures.
%% To rename them correctly to A1, A2, etc., please add the following commands in front of them:

\appendixfigures  %% needs to be added in front of appendix figures

\appendixtables   %% needs to be added in front of appendix tables

%% Please add \clearpage between each table and/or figure. Further guidelines on figures and tables can be found below.



\authorcontribution{TEXT} %% this section is mandatory

\competinginterests{TEXT} %% this section is mandatory even if you declare that no competing interests are present

\disclaimer{TEXT} %% optional section

\begin{acknowledgements}
TEXT
\end{acknowledgements}




%% REFERENCES

%% The reference list is compiled as follows:

%\begin{thebibliography}{}

%\bibitem[AUTHOR(YEAR)]{LABEL1}
%REFERENCE 1

%\bibitem[AUTHOR(YEAR)]{LABEL2}
%REFERENCE 2

%\end{thebibliography}

%% Since the Copernicus LaTeX package includes the BibTeX style file copernicus.bst,
%% authors experienced with BibTeX only have to include the following two lines:
%%
\bibliographystyle{copernicus}
\bibliography{refs_sh_impacts_aerosol.bib}
%%
%% URLs and DOIs can be entered in your BibTeX file as:
%%
%% URL = {http://www.xyz.org/~jones/idx_g.htm}
%% DOI = {10.5194/xyz}


%% LITERATURE CITATIONS
%%
%% command                        & example result
%% \citet{jones90}|               & Jones et al. (1990)
%% \citep{jones90}|               & (Jones et al., 1990)
%% \citep{jones90,jones93}|       & (Jones et al., 1990, 1993)
%% \citep[p.~32]{jones90}|        & (Jones et al., 1990, p.~32)
%% \citep[e.g.,][]{jones90}|      & (e.g., Jones et al., 1990)
%% \citep[e.g.,][p.~32]{jones90}| & (e.g., Jones et al., 1990, p.~32)
%% \citeauthor{jones90}|          & Jones et al.
%% \citeyear{jones90}|            & 1990



%% FIGURES

%% When figures and tables are placed at the end of the MS (article in one-column style), please add \clearpage
%% between bibliography and first table and/or figure as well as between each table and/or figure.

% The figure files should be labelled correctly with Arabic numerals (e.g. fig01.jpg, fig02.png).


%% ONE-COLUMN FIGURES

%%f
%\begin{figure}[t]
%\includegraphics[width=8.3cm]{FILE NAME}
%\caption{TEXT}
%\end{figure}
%
%%% TWO-COLUMN FIGURES
%
%%f
%\begin{figure*}[t]
%\includegraphics[width=12cm]{FILE NAME}
%\caption{TEXT}
%\end{figure*}
%
%
%%% TABLES
%%%
%%% The different columns must be seperated with a & command and should
%%% end with \\ to identify the column brake.
%
%%% ONE-COLUMN TABLE
%
%%t
%\begin{table}[t]
%\caption{TEXT}
%\begin{tabular}{column = lcr}
%\tophline
%
%\middlehline
%
%\bottomhline
%\end{tabular}
%\belowtable{} % Table Footnotes
%\end{table}
%
%%% TWO-COLUMN TABLE
%
%%t
%\begin{table*}[t]
%\caption{TEXT}
%\begin{tabular}{column = lcr}
%\tophline
%
%\middlehline
%
%\bottomhline
%\end{tabular}
%\belowtable{} % Table Footnotes
%\end{table*}
%
%%% LANDSCAPE TABLE
%
%%t
%\begin{sidewaystable*}[t]
%\caption{TEXT}
%\begin{tabular}{column = lcr}
%\tophline
%
%\middlehline
%
%\bottomhline
%\end{tabular}
%\belowtable{} % Table Footnotes
%\end{sidewaystable*}
%
%
%%% MATHEMATICAL EXPRESSIONS
%
%%% All papers typeset by Copernicus Publications follow the math typesetting regulations
%%% given by the IUPAC Green Book (IUPAC: Quantities, Units and Symbols in Physical Chemistry,
%%% 2nd Edn., Blackwell Science, available at: http://old.iupac.org/publications/books/gbook/green_book_2ed.pdf, 1993).
%%%
%%% Physical quantities/variables are typeset in italic font (t for time, T for Temperature)
%%% Indices which are not defined are typeset in italic font (x, y, z, a, b, c)
%%% Items/objects which are defined are typeset in roman font (Car A, Car B)
%%% Descriptions/specifications which are defined by itself are typeset in roman font (abs, rel, ref, tot, net, ice)
%%% Abbreviations from 2 letters are typeset in roman font (RH, LAI)
%%% Vectors are identified in bold italic font using \vec{x}
%%% Matrices are identified in bold roman font
%%% Multiplication signs are typeset using the LaTeX commands \times (for vector products, grids, and exponential notations) or \cdot
%%% The character * should not be applied as mutliplication sign
%
%
%%% EQUATIONS
%
%%% Single-row equation
%
%\begin{equation}
%
%\end{equation}
%
%%% Multiline equation
%
%\begin{align}
%& 3 + 5 = 8\\
%& 3 + 5 = 8\\
%& 3 + 5 = 8
%\end{align}
%
%
%%% MATRICES
%
%\begin{matrix}
%x & y & z\\
%x & y & z\\
%x & y & z\\
%\end{matrix}
%
%
%%% ALGORITHM
%
%\begin{algorithm}
%\caption{...}
%\label{a1}
%\begin{algorithmic}
%...
%\end{algorithmic}
%\end{algorithm}
%
%
%%% CHEMICAL FORMULAS AND REACTIONS
%
%%% For formulas embedded in the text, please use \chem{}
%
%%% The reaction environment creates labels including the letter R, i.e. (R1), (R2), etc.
%
%\begin{reaction}
%%% \rightarrow should be used for normal (one-way) chemical reactions
%%% \rightleftharpoons should be used for equilibria
%%% \leftrightarrow should be used for resonance structures
%\end{reaction}
%
%
%%% PHYSICAL UNITS
%%%
%%% Please use \unit{} and apply the exponential notation


\end{document}

% Old Intro

Aerosols are known to impose a net negative radiative forcing, however considerable uncertainty persists in the representation of radiative forcing due to aerosol-cloud interactions (ACI) \citep{ipcc_report_2021}. Advancements in computing power have allowed modelers to investigate key contributing factors which alter the intensity of radiative forcing due to ACI such as spatial resolution \citep{ma_how_2015}. Simultaneously, improvements in computational power have advanced the representation of aerosols in climate models through inclusion of more complex chemical mechanisms \citep{zaveri_development_2021} and detailed aerosol treatments such as the Community Aerosol and Radiation Model for Atmospheres (CARMA) , which is a 40 bin sectional model used by the Community Earth System Model 2 (CESM2) \citep{tilmes_description_2023} and has shown improvements in representing aerosol properties when compared to the 4 bin Modal Aerosol Model (MAM4). While these two key structural components to models---the treatment of aerosols and the effect of spatial resolution---have improved representation of aerosols and their radiative forcing, these aspects have yet to be explored in tandem to investigate the role of spatial heterogeneities typically below the grid scale of current climate models alongside detailed aerosol treatments in modifying the depiction of climate-relevant aerosol properties key to ACI such as cloud condensation nuclei (CCN) activity.

%Aerosol concentrations and properties vary at scales ranging from global and regional variability down to local and hyper-local spatial heterogeneity near emissions sources. Numerous lines of evidence point to the multi-scale spatial variability of aerosols. At the global-scale, satellite remote sensing with platforms such as MODIS have shown large-scale variability in bulk quantities including aerosol optical depth (\hl{cite}). At the regional scale, field campaigns have shown considerable variability in aerosol properties such as concentration, size distributions, CCN concentrations, and composition \citep{fast_using_2022}. Local in-situ measurements of aerosol properties are highly dependent on proximity to emission sources such as vehicular combustion, and past studies have shown that aerosol abundance and composition vary on the scale of 10s to 100s of meters downwind of sources due to non-linear processes such as coagulation \citep{zhu_study_2002}. Indeed, hyper-local heterogeneity in aerosol properties frustrates traditional comparison between point-source measurements and grid-cell averaged model quantities, and distributed measurement networks such as the Portable Optical Particle Spectrometer network in the Southern Great Plains (POPSnet-SGP) campaign aim to elucidate climate model uncertainty associated with sub-grid scale aerosol spatial heterogeneity \citep{asher_novel_2022}. 

%\hl{Could also note that aerosol heterogeneities are coupled to other sources of heterogeneity by inducing feedbacks on, for instance, heterogeneous surface properties such as radiative heating and soil moisture, which have been shown to induce secondary circulations (Lee et al. 2019, Tian et al. 2022)}

Aerosol-aware climate models typically contain a sub-model which governs the aerosol representation and associated processes. Given computational constraints, the aerosol treatment is often highly simplified with regard to aerosol compositional diversity. For instance, E3SM uses the modal aerosol model MAM4 which contains four lognormally distributed modes \citep{golaz_doe_2022}. Such a treatment constrains the diversity of the aerosol population to four internally-mixed modes (i.e., all particles within a mode are compositionally identical). As a result, modal and sectional aerosol treatments underrepresent the true compositional complexity of an aerosol population in which each particle ages independently. Particle-resolved aerosol models allow representation of the full compositional diversity of an aerosol population via a set of computational particles which are allowed to compositionally vary and age independently. Particle resolved models such as the Particle Monte Carlo model (PartMC) have been used extensively to investigate the sensitivity of CCN activity to the composition of emitted aerosol particles \citep{fierce_when_2013}, aging timescales due to condensation and coagulation for carbonaceous CCN \citep{fierce_explaining_2015}, and the sensitivity of CCN estimates to mixing state (i.e., the compositional diversity within and across aerosols) \citep{ching_metrics_2017}. Furthermore, given that a particle-resolved model can fully represent aerosol composition space, PartMC has been used to estimate error in CCN activity for coarser aerosol treatments by comparing depicted CCN concentrations against compositionally-averaged output reminiscent of a sectional or modal model \citep{zaveri_particle-resolved_2010, ching_metrics_2017}. Furthermore, direct comparison of CCN activity depicted by PartMC and MAM4 has shown considerable divergence, especially in polluted regions where high rates of coagulation and gas-particle partitioning amplify model disagreement \citep{fierce_quantifying_2024}. 

In addition to the treatment of aerosols, the depiction of spatial heterogeneity---including that of surface properties, emissions of gas phase species and primary aerosol, and heterogeneous distributions of their resulting plumes---influences particle aging and associated properties including CCN activity. Present aerosol-aware models at regional and global scales possess insufficient resolution to capture the full spatial heterogeneity of aerosols and the emissions of primary aerosols and gas phase precursors. In turn, these models depict artificially dilute and uniform concentrations within grid cells. This alters the representation of concentration-dependent, non-linear aerosol processes such as coagulation and gas particle partitioning. Past studies have shown that climate-relevant aerosol properties such as aerosol optical properties \citep{gustafson_jr_downscaling_2011} and CCN activity \citep{weigum_effect_2016} are highly sensitive to the model’s grid resolution with a large contribution of sub-grid scale variability resulting from the spatially varying pattern of emissions \citep{qian_investigation_2010}. 

%Given large uncertainties in the effective radiative forcing resulting from aerosol-cloud interactions \citep{ipcc_report_2021}, constraining model estimates of CCN activity remains an important focus. 

Past studies evaluating the sub-grid variability of aerosol properties often compare outputs from models with coarse resolution typical of global climate models ($\sim 50$--$100$ km) against higher resolution model outputs with resolution on the order of $\sim 1$--$10$ km 
\citep{qian_investigation_2010, gustafson_jr_downscaling_2011, weigum_effect_2016, crippa_impact_2017, lin_quantification_2017}. While such improvements in resolution better resolve heterogeneities in emissions, there still exists considerable unresolved spatial heterogeneity in the sub-kilometer scale. In addition, such modeling studies do not explicitly resolve the scales of turbulent transport in the boundary layer, instead fully parameterizing turbulence via Reynolds-averaged Navier Stokes. 

%Furthermore, these models use simplified representations of aerosols such as modal or sectional treatments. Comparing these aerosol treatments against particle-resolved models which represent the broad compositional complexity of aerosols, past studies have shown that simplified aerosol representations impact modeled aerosol properties including CCN activity \citep{zaveri_particle-resolved_2010, ching_metrics_2017, fierce_quantifying_2024}.

The evolution of emission plumes containing gas phase compounds and aerosols is highly dependent on turbulent mixing at fine spatial scales and the proximity of reactive species. A considerable body of literature has investigated the role of turbulence-chemistry interactions and chemical segregation on gas phase reactions in the planetary boundary layer via the use of large-eddy simulations (LES). \citet{brasseur_segregation_2023} review recent applications of LES to investigate chemical segregation and turbulence chemistry interactions in a variety of spatially heterogeneous geographic regions. Past studies tend to focus on the oxidation of highly reactive volatile organic compounds (VOCs) such as isoprene and have shown that spatially heterogeneous emissions contribute to the chemical segregation between reactive gas phase species in the boundary layer \citep{ouwersloot_segregation_2011, kaser_chemistry-turbulence_2015}. Note, however, that these studies do not model aerosols, although the coupling between the gas phase and aerosols through gas-particle partitioning suggests chemical segregation due to the spatial heterogeneity of emissions likely influences the aerosol state. 
%\hl{Also note Jeff Pierce's work}


Recently, numerous turbulence-resolving frameworks have been coupled to aerosol models including the use of the Sectional Aerosol model for Large Scale Applications (SALSA) \citep{kokkola_salsa_2008} alongside UCLALES \citep{tonttila_uclalessalsa_2017} and the Parallelized Large-Eddy Simulation Model (PALM) \citep{kurppa_implementation_2019} as well as the coupling between the modal model M7 \citep{vignati_m7_2004} and the Dutch Atmospheric Large-Eddy Simulation model (DALES) \citep{de_bruine_explicit_2019}.
%Both models have been used to investigate aerosol-cloud interactions and compare model outputs against field campaign measurements. 
Although these models have high-resolution transport schemes, they each possess relatively coarse-resolution aerosol treatments. For instance, UCLALES-SALSA uses a 10-bin sectional treatment while DALES implements a modified version of the seven-mode M7 model to allow two additional hydrometeor modes. To our knowledge, no aerosol-aware transport model has yet to leverage a high-resolution particle resolved aerosol treatment alongside turbulence resolving transport frameworks such as LES. 

%\hl{I don't really consider LES models that couple microphysics schemes (such as the microphysics scheme of Feingold 1996) since they simply parameterize the aerosols and number of CCN but there are more of such models in existence than purely aerosol-aware LES.}

%\begin{itemize}
%\item SAM + modal Aitken mode aerosol microphysics scheme (Wyant et al. 2022)
%\item PNNL-LES + 2D binned microphysics (Ovchinnikov and Easter 2010)
%\end{itemize}

The aim of this work is to conduct a process-level analysis of the the complex coupling between the spatial heterogeneity of surface emissions (including both gas compounds and primary aerosol), aerosol aging processes, and the resulting impact on aerosol properties including CCN activity via a set of first-of-a-kind particle-resolved LES. This establishes a high resolution aerosol-transport model framework in both explicit representation of turbulent transport as well as aerosol composition, properties, and aging.

This paper is organized in the following manner. Section 2 presents the modeling framework used in this study, WRF-PartMC-MOSAIC-LES, alongside a description of numerous emissions scenarios ranging in spatial heterogeneity. Section 3 discusses results of simulation runs, and a description of changes to the aerosol size distribution, composition, hygroscopicity, and CCN activity across emissions scenarios are discussed. We conclude with remarks on the implications of this study, limitations stemming from its idealized nature, and future work. 
