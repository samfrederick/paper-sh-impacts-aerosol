\documentclass{article}
\usepackage{amsmath,amssymb,graphicx}
\usepackage{color}
\usepackage{xcolor}
\usepackage{graphicx}
\usepackage{natbib}
\usepackage[margin=2.5cm]{geometry}
\usepackage{xr-hyper}
\usepackage{hyperref}

\usepackage[margin=2.5cm]{geometry}

\setlength{\parindent}{0em}
\setlength{\parskip}{1em}

\externaldocument{../4_revised_paper/paper_sh_impacts_aerosol}

\begin{document}

%%%%%%%%%%%%%%%%%%%%%%%%%%%%%%%%%%%%%%%%%%%%%%%%%%%%%%%%%%%%%%%%%%%%%%

\section*{Responses to Reviewer \#2}
We thank the reviewer for taking the time to review our paper and for
the constructive comments. The page and line numbers that we quote for
indicating where we changed the manuscript refer to the revised
marked-up version.

\textbf{General note:} During preparation of the revised manuscript, we identified a technical issue that required re-running all simulations. Specifically, vertical diffusion of computational particles was not correctly accounted for in the original simulations. All simulations have been rerun with this issue corrected. The revised results remain qualitatively consistent with those originally reported, and the main conclusions of the study are unchanged, although some quantitative differences appear in detailed vertical profiles.
 
%%%%%%%%%%%%%%%%%%%%%%%%%%%%%%%%%%%%%%%%%%%%%%%%%%%%%%%%%%%%%%%%%%%%%%

\textbf{(2.1)} CCN can activate below $100\%$ RH and if so, how would
this impact your results?  The paper focuses on CCN activation at low
supersaturation values and it would be interesting to determine if CCN
activation below $100\%$ RH results in similar trends or produces a
different outcome. Perhaps, since this study is highly idealized,
simulations and analysis of such a situation is not needed. However,
it could be useful to comment on this in the manuscript.

\begin{quote}
  We thank the reviewer for this question. In classical Köhler theory, CCN activation requires supersaturation with respect to water ($\mathrm{RH}>100\%$) and therefore does not occur below saturation. However, growth of particles under subsaturated conditions through hygroscopic water uptake and co-condensation of semi-volatile species can substantially modify particle size and composition prior to activation. This effect is explicitly examined in the additional high-RH simulations introduced in Section 3.8, which were added in response to Reviewer 1. We find that such pre-activation growth shifts CCN activity toward lower supersaturations and moderates, but does not eliminate, the influence of emissions spatial heterogeneity.

  We have included the following statement in Section 4, lines 424--430 acknowledging our limited treatment of CCN activity and anticipated outcomes if explicit modeling of CCN activation at subsaturated conditions were incorporated:


  ``...Furthermore, we wish to emphasize that we do not explicitly model CCN activation and droplet growth. Although previous particle-resolved studies have included this effect (Ching et al., 2012), cloud droplet growth is excluded from the current study due to computational cost. We expect that inclusion of CCN which activate at subsaturated conditions would meaningfully alter the aqueous phase chemistry within cloudy cells. Past studies have shown that in-cloud aqueous chemistry can result in the spatial segregation of reactive gas phase species, reducing oxidation of volatile organics (Li et al., 2017). We therefore hypothesize that cloud droplet aging is further coupled to emissions heterogeneity through aqueous-phase chemistry, and this coupling should be investigated in future work."
\end{quote}

\textbf{(2.2)} Could you use a profile where there’s a more humid boundary layer? I think a
boundary that would support cloud formation would be useful in that the results
would show aerosol-cloud interactions in an environment that actually forms
clouds. This would help support your explanations of what should occur given
supersaturation conditions being realized in your environment, which don’t
actually happen based on the boundary layer profile used in the experiment.

\begin{quote}
  We thank the reviewer for this recommendation. Although our model is not able to explicitly represent CCN activation and cloud microphysics, we have conducted an additional set of simulations in which the relative humidity was raised to near saturation in the upper boundary layer. Our analysis is included in Section 3.8, indicating that higher RH meaningfully impacts aerosol composition and computed CCN activity by enhancing the condensation of semi-volatile species such as ammonium and nitrate. We conclude that co-condensation lowers the sensitivity of CCN activity to emissions heterogeneity, however, meaningful impacts are still found at intermediate supersaturation. We have included an additional statement in  Section 4, lines 421--424, which acknowledges the limited applicability of WRF-PartMC's approach to computing CCN activity and calls for future work to further explore the quantitative contribution of co-condensation and emissions heterogeneity to aerosol aging and CCN activity:

  ``Our study of co-condensation and its effects reveals meaningful limitations in how WRF-PartMC-LES estimates CCN activity for particles under water-subsaturated conditions. Accordingly, CCN concentrations diagnosed under subsaturated conditions should be interpreted as an indicator of activation potential rather than a direct proxy for cloud droplet number, particularly in environments with substantial semivolatile mass. While our study reveals that emissions heterogeneity may alter aerosol composition and CCN activity even when co-condensation is considered, further work is required to place quantitative bounds on the contribution of each process and the modulating role of gas and aerosol composition.''
\end{quote}

\textbf{(2.3)} For the high heterogeneity emission scenario, is there any value in having
multiple point sources (plumes) and testing sensitivity to that configuration vs.
one point source? I’m thinking that industrial regions often have more than just
one smoke stack or concentrated emission point. Maybe this is better suited for
future work, but worth considering.

\begin{quote}
  We agree that more realistic emission patterns should be explored, and have included the following statement in Section 4, lines 408--413:

  ``Additionally, all emissions—both gas-phase and particulate—are
  temporally constant after spin-up and spatially collocated,
  reflecting a deliberate idealization used here to isolate the
  effects of emissions spatial heterogeneity. This configuration may
  not fully capture the complexity of real urban emission patterns
  where different sources (e.g., traffic, industry, biomass burning)
  are spatially and temporally decoupled. Future studies should
  investigate the impact of emissions heterogeneity on the aerosol
  state in response to realistic emission patterns such as numerous
  point sources with spatially segregated reactive species.''
\end{quote}

\textbf{(2.4)} Is there really a need to run the model at 100 m grid spacing for what this work
is trying to demonstrate? There wasn’t much of a discussion on the role of
turbulence (only briefly mentioned in the introduction) which I would assume is
important to address when running LES scale simulations. Another way of
thinking of this is, what do we learn from the LES scale simulation that is not well
represented or cannot be produced in a meso-scale simulation for this work?
This is not clearly articulated in the explanation of the model setup or in the
results.

\begin{quote}
We have included the following statement in Section 4, lines 431--438
to clarify that LES is required here to resolve emission heterogeneity
at sub-kilometer scales, rather than to study turbulence itself.

``In this study, use of LES is primarily motivated by the length scale
of heterogeneous emission patterns. The current work resolves
emissions heterogeneity down to 100 m, which is well below the grid
spacing of typical meso-scale and global models and allows explicit
representation of sharp spatial gradients in gas and aerosol
concentrations that would otherwise be averaged out. Given the
idealized configuration employed here, the resolved turbulence
primarily serves to provide realistic boundary-layer mixing under
laterally homogeneous conditions, and its detailed structure is not a
central focus of the present analysis. Nevertheless, turbulence is
expected to play an important role in modulating the interaction
between spatially segregated emissions, gas-phase chemistry, and
aerosol aging in more realistic settings. Past studies have shown that
turbulence facilitates the mixing of spatially segregated reactive gas
species, and future applications of WRF-PartMC-LES should therefore
investigate how turbulence modifies the rate and spatial structure of
aerosol aging in domains with topography, heterogeneous surface
forcing, and complex emission patterns."
\end{quote}

 
\textbf{(2.5)} Following from question 4 above, emission flux data observations are not at 100
m resolution, correct? If so, how feasible is this framework for testing against
observations for future implementation?

\begin{quote}
  We have added the following discussion on lines 413--415 to clarify
  that although gridded emission inventories are often too coarse for
  LES, key source types such as point and line sources can already be
  represented accurately at sub-kilometer scales, while diffuse area
  sources require additional preprocessing and downscaling.

  ``Emission flux observations and inventories are not uniformly
  available at 100 m resolution. While many gridded inventories used
  in regional and global-scale modeling are too coarse for direct use
  in LES, several important source types can be represented accurately
  at sub-kilometer scales. Point sources (e.g., power plants,
  industrial stacks) and line sources (e.g., road networks) are often
  well constrained spatially and can be implemented directly in
  high-resolution simulations. In contrast, diffuse area sources that
  are typically reported at county or regional scales require
  additional emission preprocessing, downscaling, or data-fusion
  approaches to distribute fluxes at finer resolution. Developing such
  preprocessing workflows represents an important step toward applying
  WRF-PartMC-LES in observationally constrained, realistic settings.''
\end{quote}

 

\textbf{(2.6)} For some of the figures, for example Figure 7, it would be nice to give the
readers a sense of the temporal variability of aerosol species mixing ratios
instead of just at t=6 h. I wouldn’t say this is a must, but it would be helpful to
see.

\begin{quote}
  We thank the reviewer for this recommendation. In response to this suggestion and another reviewer's recommendation to include simulations where emissions are turned off after a certain period of time, we have included Figure 13 (included here for convenience). This figure compares the evolution of aerosol composition for both the high heterogeneity scenario in which emissions continue unabated and a similar scenario but with emissions turned off at $t=4$ h. Vertical profiles are plotted in hourly increments to illustrate the temporal evolution of the aerosol state.
\end{quote}

\begin{figure}[!h]
	\centering
	\includegraphics[]{../4_revised_paper/figures/high-het-relaxation.pdf}
	\caption{Figure 13 in Frederick et al., 2026: ``Vertical profiles of ammonium (top row) and nitrate (bottom row) for two versions of the high heterogeneity scenario in which emissions are turned off at $t=4$~h (left column) and emissions continue through the remainder of the simulation (right column)."}
	\label{fig:nh4-no3-relaxation}
\end{figure}

\textbf{(2.7)} Typo line 85: “it is extends"

\begin{quote}
  We have fixed this typo. 
\end{quote}

\textbf{(2.8)} Typo line 94: “by by”

\begin{quote}
  We have fixed this typo. 
\end{quote}

\textbf{(2.9)} Repeated sentences around line 138

\begin{quote}
  We have removed the duplicate sentence.  
\end{quote}


%\bibliographystyle{copernicus}
%\bibliography{../4_revised_paper/refs_tchem_paper.bib}

\end{document}
