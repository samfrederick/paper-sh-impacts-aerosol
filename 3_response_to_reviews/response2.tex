\documentclass{article}
\usepackage{amsmath,amssymb,graphicx}
\usepackage{color}
\usepackage{xcolor}
\usepackage{graphicx}
\usepackage{natbib}
\usepackage[margin=2.5cm]{geometry}
\usepackage{xr-hyper}
\usepackage{hyperref}

\usepackage[margin=2.5cm]{geometry}

\setlength{\parindent}{0em}
\setlength{\parskip}{1em}

\externaldocument{../4_revised_paper/paper_sh_impacts_aerosol}

\begin{document}

%%%%%%%%%%%%%%%%%%%%%%%%%%%%%%%%%%%%%%%%%%%%%%%%%%%%%%%%%%%%%%%%%%%%%%

\section*{Responses to Reviewer \#2}
We thank the reviewer for taking the time to review our paper and for
the constructive comments. The page and line numbers that we quote for
indicating where we changed the manuscript refer to the revised
marked-up version.
 
%%%%%%%%%%%%%%%%%%%%%%%%%%%%%%%%%%%%%%%%%%%%%%%%%%%%%%%%%%%%%%%%%%%%%%

\textbf{(1.1)} CCN can activate below $100\%$ RH and if so, how would this impact your results?
The paper focuses on CCN activation at low supersaturation values and it would
be interesting to determine if CCN activation below $100\%$ RH results in similar
trends or produces a different outcome. Perhaps, since this study is highly
idealized, simulations and analysis of such a situation is not needed. However, it
could be useful to comment on this in the manuscript.

\begin{quote}
  We have included the following statement in Section \ref{sec:limitations} (Limitations and future work) acknowledging our limited treatment of CCN activity and anticipated outcomes if explicit modeling of CCN activation at subsaturated conditions were incorporated:


  ``...Furthermore, we wish to emphasize that we do not model explicitly model CCN activation and droplet growth. Although previous particle-resolved studies have included this effect \citep{ching_impacts_2012}, cloud droplet growth is excluded from the current study due to computational cost. We expect that inclusion of CCN which activate at subsaturated conditions would meaningfully alter the aqueous phase chemistry within cloudy cells. Past studies have shown that in-cloud aqueous chemistry can result in the spatial segregation of reactive gas phase species, reducing oxidation of volatile organics \citep{li_impact_2017}. We therefore hypothesize that cloud droplet aging is further coupled to emissions heterogeneity through aqueous-phase chemistry, and this coupling should be investigated in future work."
\end{quote}

\textbf{(1.2)} Could you use a profile where there’s a more humid boundary layer? I think a
boundary that would support cloud formation would be useful in that the results
would show aerosol-cloud interactions in an environment that actually forms
clouds. This would help support your explanations of what should occur given
supersaturation conditions being realized in your environment, which don’t
actually happen based on the boundary layer profile used in the experiment.

\begin{quote}
  We thank the reviewer for this recommendation. Although our model is not able to explicitly represent CCN activation and cloud microphysics, we have conducted an additional set of simulations in which the relative humidity was raised to near saturation in the upper boundary layer. Our analysis is included in Section \ref{sec:co-condensation}, indicating that higher RH meaningfully impacts aerosol composition and computed CCN activity by enhancing the condensation of semi-volatile species such as ammonium and nitrate. We conclude that co-condensation lowers the sensitivity of CCN activity to emissions heterogeneity, however, meaningful impacts are still found at intermediate supersaturation. We have included an additional statement in  Section \ref{sec:limitations} (Limitations and future work), which acknowledges the limited applicability of WRF-PartMC's approach to computing CCN activity and calls for future work to further explore the quantitative contribution of co-condensation and emissions heterogeneity to aerosol aging and CCN activity. 
\end{quote}

\textbf{(1.3)} For the high heterogeneity emission scenario, is there any value in having
multiple point sources (plumes) and testing sensitivity to that configuration vs.
one point source? I’m thinking that industrial regions often have more than just
one smoke stack or concentrated emission point. Maybe this is better suited for
future work, but worth considering.

\begin{quote}
  We agree that more realistic emission patterns should be explored, and have included the following statement in Section \ref{sec:limitations} (Limitations and future work):

  ``Additionally,
all emissions—both gas-phase and particulate—are temporally constant
after spin-up and spatially collocated, which may not reflect the
complexity of real urban emission patterns where different sources
(e.g., traffic, industry, biomass burning) are spatially and
temporally decoupled. Future studies should investigate the impact of emissions heterogeneity on the aerosol state in response to realistic emission patterns such as numerous point sources with spatially segregated reactive species.''
\end{quote}

\textbf{(1.4)} Is there really a need to run the model at 100 m grid spacing for what this work
is trying to demonstrate? There wasn’t much of a discussion on the role of
turbulence (only briefly mentioned in the introduction) which I would assume is
important to address when running LES scale simulations. Another way of
thinking of this is, what do we learn from the LES scale simulation that is not well
represented or cannot be produced in a meso-scale simulation for this work?
This is not clearly articulated in the explanation of the model setup or in the
results.

\begin{quote}
We have included the following statement in Section \ref{sec:limitations} (Limitations and future work) to clarify that the motivation for the use of LES is primarily due to the length scale of emissions heterogeneity we wish to model. We acknowledge that future studies should extend upon this work to investigate the role of turbulence in modulating the impact of emissions heterogeneity in more realistic domains:

``In this study, use of LES is primarily motivated by the length scale of heterogeneous emission patterns. The current work resolves emissions heterogeneity down to 100 m, significantly below regional and global-scale model resolutions. 
The idealized nature of the current work means there is limited benefit of grid-resolved turbulence as the structure of turbulence the boundary layer is mostly laterally isotropic. However, greater consideration should be given to resolved turbulence in future applications of WRF-PartMC-LES, especially when modeling realistic domains with topography, varied land use,  heterogeneous surface heat fluxes, and spatially segregated emissions. Past studies have shown that turbulence facilitates the mixing of spatially segregated reactive gas species. Given the coupling between the gas and aerosol phase, future work should investigate the role of turbulence in modifying the rate of aerosol aging."  
\end{quote}

 
\textbf{(1.5)} Following from question 4 above, emission flux data observations are not at 100
m resolution, correct? If so, how feasible is this framework for testing against
observations for future implementation?

\begin{quote}
  We have added the following discussion on \textcolor{red}{lines X to X}:
  ``...gridded inventories commonly used for regional and global-scale modeling are too coarse for use with LES. Future extensions of WRF-PartMC-LES to realistic emission patterns will require use of data fusion techniques to estimate source apportionment at sub-kilometer resolution.''
\end{quote}

 

\textbf{(1.6)} For some of the figures, for example Figure 7, it would be nice to give the
readers a sense of the temporal variability of aerosol species mixing ratios
instead of just at t=6 h. I wouldn’t say this is a must, but it would be helpful to
see.

\begin{quote}
  We thank the reviewer for this recommendation. In response to this suggestion and another reviewer's recommendation to include simulations where emissions are turned off after a certain period of time, we have included Figure \ref{fig:nh4-no3-relaxation}. This figure compares the evolution of aerosol composition for both the high heterogeneity scenario in which emissions continue unabated and a similar scenario but with emissions turned off at $t=4$ h. Vertical profiles are plotted in hourly increments to illustrate the temporal evolution of the aerosol state.
\end{quote}

\textbf{(1.7)} Typo line 85: “it is extends"

\begin{quote}
  We have fixed this typo. 
\end{quote}

\textbf{(1.8)} Typo line 94: “by by”

\begin{quote}
  We have fixed this typo. 
\end{quote}

\textbf{(1.9)} Repeated sentences around line 138

\begin{quote}
  We have removed the duplicate sentence.  
\end{quote}


%\bibliographystyle{copernicus}
%\bibliography{../4_revised_paper/refs_tchem_paper.bib}

\end{document}