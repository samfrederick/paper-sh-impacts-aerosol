\documentclass{article}
\usepackage{amsmath,amssymb,graphicx}
\usepackage{color}
\usepackage{xcolor}
\usepackage{graphicx}
\usepackage{natbib}

\usepackage{xr-hyper}
\usepackage{hyperref}

\usepackage[margin=2.5cm]{geometry}

\setlength{\parindent}{0em}
\setlength{\parskip}{1em}

\externaldocument{../4_revised_paper/paper_sh_impacts_aerosol}

\begin{document}

%%%%%%%%%%%%%%%%%%%%%%%%%%%%%%%%%%%%%%%%%%%%%%%%%%%%%%%%%%%%%%%%%%%%%%

\section*{Responses to Reviewer \#1}
We thank the reviewer for taking the time to review our paper and for
the constructive comments. The page and line numbers that we quote for
indicating where we changed the manuscript refer to the revised
marked-up version.

\textbf{General note:} During preparation of the revised manuscript, we identified a technical issue that required re-running all simulations. Specifically, vertical diffusion of computational particles was not correctly accounted for in the original simulations. All simulations have been rerun with this issue corrected. The revised results remain qualitatively consistent with those originally reported, and the main conclusions of the study are unchanged, although some quantitative differences appear in detailed vertical profiles.
 
%%%%%%%%%%%%%%%%%%%%%%%%%%%%%%%%%%%%%%%%%%%%%%%%%%%%%%%%%%%%%%%%%%%%%%

\textbf{(1.1)} One issue in the presented analysis is its connection to aerosol-cloud interactions.
Defining CCN properties at low relative humidity (RH) does not account for the
continued condensation of semivolatile compounds at higher RH. A portion of these
compounds would condense below $100\%$ RH, thereby altering the CCN distribution.
Furthermore, most nitric acid would condense onto particles prior to activation into
cloud droplets, enhancing droplet formation. Including this effect could significantly
change the number of cloud droplets formed. This could potentially even reverse the
observation presented in this study. Therefore, using CCN as a proxy for cloud droplet
concentration in scenarios with a strong contribution form semivolatile aerosol
compounds may be somewhat misleading.

\begin{quote}
  We thank the reviewer for identifying this limitation with our analysis. In order to address the effects of co-condensation, we have conducted an additional set of simulations in which the relative humidity was raised to near saturation in the upper boundary layer. Our analysis is included in Section 3.8, indicating that higher RH meaningfully impacts aerosol composition and diagnosed CCN activity by enhancing the condensation of semi-volatile species such as ammonium and nitrate. We conclude that co-condensation lowers the sensitivity of CCN activity to emissions heterogeneity, however, meaningful impacts persist at intermediate supersaturation. 
  
  We have included the following statement on lines 421--430, which acknowledges the limited applicability of WRF-PartMC's approach to computing CCN activity and calls for future work to further explore the quantitative contribution of co-condensation and emissions heterogeneity to aerosol aging and CCN activity:

  ``Our study of co-condensation and its effects reveals meaningful limitations in how WRF-PartMC-LES estimates CCN activity for particles under water-subsaturated conditions. Accordingly, CCN concentrations diagnosed under subsaturated conditions should be interpreted as an indicator of activation potential rather than a direct proxy for cloud droplet number, particularly in environments with substantial semivolatile mass. While our study reveals that emissions heterogeneity may alter aerosol composition and CCN activity even when co-condensation is considered, further work is required to place quantitative bounds on the contribution of each process and the modulating role of gas and aerosol composition. Furthermore, we wish to emphasize that we do not model explicitly CCN activation and droplet growth. Although previous particle-resolved studies have included this effect (Ching et al., 2012), cloud droplet growth is excluded from the current study due to computational cost. We expect that explicitly modeling CCN activation and subsequent droplet growth would meaningfully alter aqueous-phase chemistry within cloudy cells.  Past studies have shown that in-cloud aqueous chemistry can result in the spatial segregation of reactive gas phase species, reducing oxidation of volatile organics (Li et al., 2017). We therefore hypothesize that cloud droplet aging is further coupled to emissions heterogeneity through aqueous-phase chemistry, and this coupling should be investigated in future work.''
\end{quote}

\textbf{(1.2)} What would happen if a Lagrangian perspective were adopted, assuming air masses
advect over the emission source? With the modeling framework used here, this could
have been explored by halting emissions from the point source partway through the
simulation and allowing the emitted compounds to disperse within the domain. Would
the effects observed in Figure 6 and beyond be averaged out due to the reversible
nature of nitrate partitioning? Such a setup would more closely reflect the assumptions
made in low-resolution, large-scale models.

\begin{quote}
  We thank the reviewer for this recommendation. We have conducted an additional simulation in which emissions are cut off at $t=4$ h to investigate the reversible partitioning of nitrate. Discussion of this scenario is included in Section 3.7. As expected, the most significant impact to aerosol composition is due to reversible partitioning of nitrate. The timescale at which the aerosol state relaxes back to the aerosol composition in the no heterogeneity scenario is therefore determined by the timescale of reversible partitioning for nitrate.

  We find that the effects of emissions heterogeneity are progressively reduced after emissions cease, but are not instantaneously averaged out; instead, the relaxation timescale is governed by the timescale of nitrate repartitioning and boundary-layer mixing.
\end{quote}

\textbf{(1.3)} Line 50: The statement “yet many climate models fail to resolve this variability
adequately” could be clarified. Are there actually any climate models that attempt to
account for subgrid-scale heterogeneity in a proper manner for the emissions?

\begin{quote}
  We have included the following statement on lines 49--52 to clarify the status of sub-grid scale parameterizations in global scale models:

  ``Much of the sub-grid scale variability arises from
  spatially heterogeneous emissions (Qian et al., 2010), yet most climate models do not represent this variability explicitly in a general sense. Existing parameterizations in global climate models that account for sub-grid scale emissions and associated processes are typically limited to specific phenomena, such as ship tracks (Huszar et al., 2010) and contrails (Burkhardt and Kärcher, 2009), rather than providing a general treatment of emissions-driven aerosol heterogeneity.''

\end{quote}

\textbf{(1.4)} Line 71--72: For precision, note that SALSA by default uses a 17-bin scheme (10 + 7), similar
to how M7 employs 7 modes (4 + 3) to represent externally mixed aerosol populations
with high and low hygroscopicities. 

\begin{quote}
  Thank you for this point of clarification. We have revised discussion of SALSA on line 72 to the following: ``\dots UCLALES-SALSA employs a
17-bin sectional scheme to represent externally mixed
aerosol populations with differing hygroscopicities, \dots'' 
\end{quote}

\textbf{(1.5)} Line 85: Typo: “It is extends”

\begin{quote}
  We have fixed this typo. 
\end{quote}

\textbf{(1.6)} Line 94: Typo: “by by”

\begin{quote}
  We have fixed this typo. 
\end{quote}

\textbf{(1.7)} Line 130: In large-eddy simulation (LES) studies, heat flux is more commonly expressed
in W/m² rather than Km/s.

\begin{quote}
  Thank you for noting this convention. We have converted the surface heat flux to non-kinetmatic form and note its relevance to mid-latitude settings on lines 132--133:
  ``...295.5~$\rm W \, m^2$, which is
representative of surface heat fluxes resulting from annually-averaged mid-latitude solar insolation.''

\end{quote}

\textbf{(1.8)} Lines 137–140: The same sentence appears to be repeated. Please remove the
duplicate.

\begin{quote}
  We have removed the duplicate sentence. 
\end{quote}

\textbf{(1.9)} Figure 4: The concentrations of nitric acid and ammonia seem quite high. Are these
values realistic, or do they represent an extreme scenario? A brief discussion on this
would be needed.

\begin{quote}
  We have added the following statement on lines 215--217 to clarify that the concentrations modeled in this study correspond to the upper end of observed concentrations in highly-polluted conditions:

``Comparing modeled ammonium and nitric acid levels with past studies,
  the concentrations used here correspond to the upper end of observed
  values reported during brief, extremely polluted episodes in regions
  with substantial vehicular emissions, such as Southern California
  (Salmon et al., 1990; Toro et al., 2024)."


\end{quote}

\textbf{(1.10)} Figure 5: If new particle formation via nucleation is not included in the study, could this
omission influence the results?

\begin{quote}
  We have included the following statement on lines 439--446 to clarify the omission of nucleation in WRF-PartMC-LES:
  
  ``Lastly, nucleation is not modeled in WRF-PartMC-LES due to the
  large computational expense associated with explicitly representing
  the coagulation of large numbers of ultrafine computational
  particles. As a result, ultrafine particle concentrations may be
  underestimated in the present simulations. Importantly, nucleation
  itself is highly sensitive to spatial heterogeneity, as nucleation
  rates depend nonlinearly on precursor concentrations and local
  thermodynamic conditions. Spatial averaging over coarse grid cells
  can therefore suppress the peak conditions required for nucleation,
  leading to an underprediction of nucleation events even in models
  that nominally include this process. The inclusion of nucleation
  could influence the results shown here, particularly through its
  interaction with spatially heterogeneous emissions and gas-phase
  precursor fields. In highly heterogeneous scenarios, enhanced
  nucleation rates in localized regions may be partially offset by
  increased coagulation with larger particles, whereas in
  lower-heterogeneity environments reduced coagulation could allow a
  more pronounced ultrafine mode to persist. The net impact of
  nucleation on the size distribution and CCN activity therefore
  depends on the relative rates of nucleation, coagulation, and
  mixing, as well as on the degree of emissions heterogeneity. As a
  result, inclusion of nucleation could either dampen or amplify
  differences in CCN activity between low- and high-heterogeneity
  cases, and explicitly resolving these interactions remains an
  important topic for future work.''

\end{quote}


%\bibliographystyle{copernicus}
%\bibliography{../4_revised_paper/refs_tchem_paper.bib}

\end{document}
